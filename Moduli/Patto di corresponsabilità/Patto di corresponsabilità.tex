% !TeX spellcheck = it_IT
% !TeX encoding = UTF-8
\documentclass{djtsmod}

\title{PATTO DI CORRESPONSABILITÀ EDUCATIVO}

\begin{document}
	\maketitle
	
	\begin{center}\bfseries
		tra l'ASD DOJO Trieste ed esercenti la responsabilità genitoriale, tutori e affidatari degli allievi minori \\[5pt]
		ai sensi dell'art. 3, lett. g, punto 4 del Modello Organizzativo (di Gestione) e di Controllo dell'attività sportiva avente ad oggetto "Obbligo di Vigilanza negli spogliatoi"
	\end{center}
	
	\statement{PREMESSA}
	Il Patto educativo di corresponsabilità:
	\begin{enumerate}[label=\alph*)]
		\item è il documento che deve essere firmato dagli esercenti la potestà genitoriale, tutori e affidatari degli allievi minori che praticano attività sportiva con l'ASD Dojo Trieste e che enuclea i principi e i comportamenti che l'ASD, gli esercenti la potestà genitoriale, tutori e affidatari degli allievi minori condividono e si impegnano a rispettare e a far rispettare per la tutela dei minori e per la prevenzione delle molestie, della violenza di genere e di ogni altra condizione di discriminazione (c.d. politiche di Safeguarding);
		\item è il documento che, coinvolgendo tutti i soggetti che gravitano attorno ai minori che svolgono attività sportiva all'interno dell'ASD Dojo Trieste, si presenta come strumento base dell'interazione ASD/nucleo familiare e/o affidatario, tutore al fine di perseguire una comune azione educativa, nel rispetto di ruoli e di responsabilità specifiche dei singoli soggetti;
		\item il testo del patto di responsabilità è stato approvato dal Consiglio Direttivo dell'ASD DOJO Trieste in data $\ldots\ldots\ldots$ ed è applicabile immediatamente dopo la sua approvazione;
		\item il testo del patto di responsabilità attiene, in particolare, l'accudimento e la sorveglianza negli spogliatoi dei minori di età inferiore agli anni 12 in considerazione della capacità dei minori tra i 12 e i 14 anni di gestire in autonomia le fasi del cambio abiti.
	\end{enumerate}
	Su queste premesse, considerato che il MOG adottato dall'ASD DOJO Trieste, prevede espressamente che: "gli accompagnatori dei minori sosteranno nei corridoi senza entrare negli spogliatoi. Soltanto per i preagonisti delle categorie Bambini/e A (4 e 5 anni), Bambini/e B (6 e 7 anni), Fanciulli/e (8 e 9 anni) e Ragazzi/e (10 e 11) anni sarà permesso l'ingresso negli spogliatoi di un massimo di due esercenti la responsabilità genitoriale o tutoria ovvero preposti alla vigilanza, scelti fra tutti gli appartenenti a dette categorie. I soggetti designati, anche con turnazione, dovranno essere preferibilmente dello stesso sesso dei minori. I soggetti designati potranno sostare negli spogliatoi solo per il tempo necessario del cambio e non potranno rimanere negli spogliatoi durante le lezioni." \\[5pt]
	Gli esercenti la responsabilità genitoriale o tutoria e gli affidatari, anche per conto degli accompagnatori delegati
	\tinystatement{DICHIARANO}
	\begin{enumerate}[label=\alph*)]
		\item di condividere la scelta di permettere la presenza di solo due adulti negli spogliatoi per l'accudimento e la sorveglianza dei minori di età inferiore agli anni 12 in considerazione della capacità dei minori tra i 12 e i 14 anni di gestire in autonomia le fasi del cambio abiti;
		\item di condividere la scelta che le due persone richieste per vigilare negli spogliatoi potranno essere individuate tra tutti gli esercenti la responsabilità genitoriale o tutoria, gli affidatari, ovvero coloro preposti alla vigilanza (da intendersi come accompagnatori autorizzati, baby sitter et similia); 
		\item di condividere la scelta che le due persone preposte alla vigilanza possano essere variabili per ogni giornata di allenamento, al fine di garantire sempre una adeguata sorveglianza dei minori negli spogliatoi.di concordare che i soggetti individuati come al punto b) dovranno essere preferibilmente dello stesso sesso dei minori che andranno ad accudire;
		\item di impegnarsi, di concerto con gli istruttori della ASD, ad illustrare ai minori il compito e le funzioni degli adulti che saranno ammessi negli spogliatoi dei minori;
		\item di impegnarsi a partecipare alle riunioni che saranno eventualmente organizzate dall'ASD con il Responsabile del Safeguarding per verificare la rispondenza alle esigenze di tutela dei minori dell'iniziativa;
		\item di segnalare qualsiasi comportamento inopportuno, ai sensi del codice di condotta e del MOG adottati dalla ASD, da parte degli adulti che saranno ammessi negli spogliatoi dei minori al Responsabile del Safeguarding;
		\item di segnalare qualsiasi comportamento inopportuno, ai sensi del codice di condotta e del MOG adottati dalla ASD, da parte dei minori all'interno degli spogliatoi.
	\end{enumerate}
	
	\datesignfield{Il Presidente della ASD DOJO Trieste}
	
	\newpage
	
	Il/la sottoscritto/a \fieldfill\; nella qualità di\footnote{L'esercente la potestà genitoriale, il tutore o l'affidatario deve essere il soggetto indicato nell'art. 4, comma 8 dello Statuto associativo, ovvero colui/colei che ha sottoscritto la domanda di ammissione a socio.}
	\begin{radiolist}
		\item esercente la responsabilità genitoriale  
		\item tutore o affidatario
	\end{radiolist}
	del minore / dei minori: \\
	\begin{minipage}{0.8\textwidth}
		\begin{enumerate}
			\item Cognome e Nome: \fieldfill
			\item Cognome e Nome: \fieldfill
			\item Cognome e Nome: \fieldfill
		\end{enumerate}
	\end{minipage}\\[20pt]
	dichiara di aver preso visione di quanto sopra riportato e di condividere in pieno gli obiettivi e gli impegni.
	
	\vspace{0.5in}
	\datesignfield{Firma}
	
\end{document}