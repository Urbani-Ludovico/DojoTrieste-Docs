% !TeX spellcheck = it_IT
% !TeX encoding = UTF-8
\documentclass{djts}

\usepackage[
	margin = 0.6in,
	bottom = 1.8in
]{geometry}

\setlist[itemize]{noitemsep, topsep = 4pt}
\setlist[enumerate]{noitemsep, topsep = 4pt}

\usepackage{fancyhdr}
\pagestyle{fancy}
\fancyhf{}
\chead{%
	\includegraphics[height = 1in]{logo_fijlkam_h.png}%
}
\setlength\headheight{1.2in}
\renewcommand{\headrulewidth}{0pt}


\usepackage{sectsty}
\sectionfont{\large\selectfont}
\renewcommand{\thesection}{\arabic{section}.}

\renewcommand{\thesubsection}{\Alph{subsection}.}


\makeatletter
\renewcommand{\maketitle}{%
	\begin{center}%
		\textbf{\large \@title}%
	\end{center}%
}
\makeatother
\title{INFORMATIVA SUL TRATTAMENTO DEI DATI PERSONALI}

\begin{document}
	\begin{flushright}
		\textbf{AN/Mod. AT15/2024}
	\end{flushright}
	\vspace{0.1in}
	
	\maketitle
	
	La presente informativa è resa nel rispetto dell'art. 13 del Regolamento UE 2016/679, relativo alla protezione delle persone fisiche con riguardo al trattamento dei dati personali, nonché alla libera circolazione di tali dati (di seguito, il "Regolamento" o il "GDPR"), dalla Federazione Italiana Judo Lotta Karate Arti Marziali, con sede in Via dei Sandolini, 79 – 00122 Ostia Lido (RM) (di seguito anche solo il "Titolare" o la "Federazione"). \\
	In questa informativa le illustreremo le finalità e le modalità con cui il Titolare raccoglie e tratta i suoi dati personali ai fini del rapporto di tesseramento, quali categorie di dati sono oggetto di trattamento, quali sono i diritti degli interessati al trattamento e come possono essere esercitati.
	
	\section{Categorie di dati personali e finalità del trattamento}
	La Federazione raccoglierà tutti i dati personali necessari al tesseramento e allo svolgimento delle attività federali, quali:
	\begin{enumerate}[label=\alph{*}.]
		\item dati anagrafici (nome, cognome, luogo e data di nascita, cittadinanza);
		\item altri dati comuni (numero di telefono, indirizzo di residenza, indirizzo di posta elettronica);
		\item nonché immagini fotografiche e filmiche (incluse registrazioni durante le sessioni di allenamento e nel corso di eventi e incontri sportivi presso i centri sportivi federali o in occasioni di manifestazioni pubbliche);
		\item dati di natura particolare ai sensi dell'art. 9 GDPR, ai fini del corretto svolgimento del rapporto di tesseramento (ad esempio, lo stato generale di salute e la relativa idoneità o meno allo svolgimento di determinate attività; informazioni sulle festività religiose o sulle preferenze alimentari; informazioni attestanti eventuali disabilità).
	\end{enumerate}
	La informiamo che i dati saranno trattati in forma cartacea e con il supporto di strumenti elettronici/informatici.
	
	\section{Finalità del trattamento e basi giuridiche}
	
	\subsection{Rapporto di tesseramento}
	In virtù del rapporto di tesseramento, i dati raccolti saranno utilizzati per le seguenti finalità, strettamente connesse alle attività federali:
	\begin{enumerate}[label=\alph{*})]
		\item gestione amministrativa e assicurativa dei tesserati;
		\item inserimento nei sistemi federali, anche ai fini delle anagrafiche dei tesserati;
		\item programmazione e organizzazione, anche logistica, delle attività e degli eventi sportivi;
		\item reclutamento, selezione, valutazione degli atleti;
		\item finalità analitiche e di studio sulle prestazioni e sugli incontri sportivi, finalità storico-archivistiche;
		\item difesa dei diritti della Federazione.
	\end{enumerate}
	Per le attività di trattamento sopra menzionate, riconducibili al rapporto di tesseramento, la base giuridica individuata è l'esecuzione di un contratto di cui è parte l'interessato [art. 6, par. 1, lett. b) GDPR], nonché l'adempimento di un obbligo di legge [art. 6, par. 1, lett. c) GDPR], nonché, per quanto concerne i dati appartenenti a categorie particolari, anche per lo svolgimento di compiti di rilevante interesse pubblico [art. 9, par. 2, lett. g) GDPR].
	
	\subsection{Finalità istituzionali della Federazione}
	I suoi dati saranno altresì trattati conformemente alla disciplina dell'ordinamento sportivo e potranno essere altresì oggetto di pubblicazione sul sito istituzionale della Federazione sotto forma di provvedimenti adottati dagli Organi di Giustizia Sportiva Federale o di altri organismi, anche internazionali, di giustizia sportiva al fine di garantire:
	\begin{enumerate}[label=\alph{*})]
		\item il corretto svolgimento delle competizioni sportive;
		\item la decorrenza dei termini d'impugnazione;
		\item nonché l'eventuale valutazione della recidiva dell'interessato tesserato.
	\end{enumerate}
	In tal caso, i trattamenti relativi si fondano sull'esecuzione di un compito d'interesse pubblico [art. 6, par. 1, lett. e) GDPR], nonché sul legittimo interesse della Federazione alla corretta gestione del contenzioso, ad ottemperare alle prescrizioni provenienti dal CONI e dai suoi regolamenti, oltre che l'eventuale pubblicazione delle immagini delle competizioni sportive, se di interesse pubblico, che prevale sui diritti dell'interessato coinvolto dal provvedimento citato [art. 6, par. 1, lett. f) GDPR].
	
	\subsection{Attività di marketing}
	Previo suo consenso specifico e facoltativo, i suoi dati saranno trattati per finalità di marketing, ovvero per l'invio di comunicazioni commerciali attraverso sistemi tradizionali (ad es. posta cartacea) e automatizzati (ad es. e-mail, SMS, notifiche), nonché per lo svolgimento di attività pubblicitarie e di sponsorizzazione di eventi, tornei e manifestazioni sportive. \\
	Il consenso prestato per le suddette finalità di marketing è revocabile in qualsiasi momento scrivendo a \href{mailto:dpo@fijlkam.it}{dpo@fijlkam.it}; in ogni caso, la revoca effettuata non pregiudica la legittimità dei trattamenti effettuati prima del suo intervento (art. 7, par. 3 GDPR).
	
	\subsection{Cessione a terzi}
	Previo suo consenso specifico e facoltativo, la Federazione potrà comunicare alcuni suoi dati
	personali a terzi (società del settore, sponsor della Federazione), che potranno trattarli ai fini dell'invio di comunicazioni commerciali attraverso sistemi tradizionali (ad es. posta cartacea) e automatizzati (ad es. e-mail, SMS, notifiche). \\
	Il consenso prestato per le suddette finalità di cessione a terzi è revocabile in qualsiasi momento scrivendo a \href{mailto:dpo@fijlkam.it}{dpo@fijlkam.it}; in ogni caso, la revoca effettuata non pregiudica la legittimità dei trattamenti effettuati prima del suo intervento (art. 7, par. 3 GDPR).
	
	\section{Categorie di destinatari}
	Ferme restando le comunicazioni eseguite in adempimento di obblighi di legge cui è soggetto il Titolare,
	tutti i dati raccolti ed elaborati potranno essere comunicati a:
	\begin{itemize}[label=-]
		\item Comitato Olimpico Nazionale Italiano (CONI), Sport e Salute S.p.a.; Organismi Antidoping dell'Ordinamento sportivo; Associazioni e Società sportive; Enti e/o altre Federazioni sportive italiane ed estere; altri soggetti pubblici o privati e organismi associativi, per la realizzazione di tutte le attività e iniziative connesse ai fini istituzionali e di promozione sportiva della Federazione;
		\item Enti, Società o altri soggetti che intrattengono rapporti contrattuali con la Federazione, ai fini dell'organizzazione e gestione, anche logistica, degli eventi sportivi e istituzionali della Federazione, nonché ai fini dello svolgimento di attività pubblicitarie, promozionali o di sponsorizzazione di eventi federali e dell'attività sportiva in generale;
		\item Società del settore, sponsor della Federazione, ai quali i suoi dati saranno comunicati previo suo esplicito, libero e facoltativo consenso, per finalità di promozione commerciale, così come meglio descritte al paragrafo precedente, alla lettera D.
		\item Imprese Assicuratrici; Consulenti esterni della Federazione, nei limiti di quanto necessario allo svolgimento del proprio mandato, ai fini di gestione amministrativa, contabile, fiscale della Federazione (a titolo esemplificativo e non esaustivo: Società di sviluppo, gestione e manutenzione di sistemi informatici ed elaborazione dati; studi e professionisti per la consulenza legale; Società di consulenza fiscale, amministrativa e contabile; personale sanitario incaricato dalla Federazione; Organi di giustizia sportiva; ecc.).
	\end{itemize}
	Tali soggetti tratteranno i suoi dati come autonomi titolari del trattamento, ovvero, laddove richiesto dalla natura del rapporto con la Federazione e nel pieno rispetto della disciplina applicabile, come responsabili del trattamento espressamente individuati per iscritto, ai quali sono fornite specifiche istruzioni circa il trattamento di dati personali. L'elenco dei responsabili del trattamento può essere domandato scrivendo	alla Federazione. \\
	Nell'ambito delle operazioni di trattamento, inoltre, potranno venire a conoscenza dei suoi dati le seguenti categorie di persone autorizzate a compiere specifiche operazioni di trattamento, sotto l'autorità e l'organizzazione del Titolare, e debitamente istruite a tal fine: membri degli organi federali; dipendenti e collaboratori a vario titolo della Federazione.
	
	\section{Diritti dell'interessato}
	La informiamo che, in conformità alla vigente disciplina, ha i seguenti diritti:
	\begin{itemize}[label=-]
		\item Accesso (art. 15 GDPR): diritto di ottenere dal Titolare la conferma che sia o meno in corso un trattamento di suoi dati personali e, in caso affermativo, di ottenere l'accesso a tali dati e alle informazioni sul trattamento;
		\item Rettifica (art. 16 GDPR): diritto a che i suoi dati personali trattati dal Titolare siano corretti e quindi, se necessario, aggiornati;
		\item Cancellazione (art. 17 GDPR): diritto di ottenere che i suoi dati personali nella disponibilità del Titolare siano cancellati al ricorrere di determinate circostanze (ad es. al raggiungimento di tutte le finalità per cui i dati erano stati raccolti);
		\item Limitazione del trattamento (art. 18 GDPR): diritto di ottenere dal Titolare che il trattamento dei suoi dati personali sia limitato al ricorrere di determinate circostanze (ad es. quando l'interessato contesta l'esattezza dei dati trattati o la prevalenza di un legittimo interesse del Titolare su cui si fonda il trattamento);
		\item Opposizione (art. 21): diritto di opporsi in qualsiasi momento, per motivi connessi alla sua situazione particolare, al trattamento di suoi dati personali, se questo è fondato sul legittimo interesse del Titolare oppure è svolto in esecuzione di un compito di interesse pubblico;
		\item Processo decisionale automatizzato (art. 22 GDPR): diritto a non essere sottoposti a decisioni basate su di un trattamento interamente automatizzato di dati personali (ad es. mediante intelligenza artificiale e senza intervento umano), compresa la profilazione, laddove tale decisione produca effetti che incidono significativamente sulla sua persona.
	\end{itemize}
	Per esercitare i diritti dell'interessato, è possibile contattare il Titolare scrivendo a Federazione Italiana Judo Lotta Karate Arti Marziali, con sede in Via dei Sandolini, 79 – 00122 Ostia Lido (RM), contattabile al seguente indirizzo di posta elettronica: \href{mailto:dpo@fijlkam.it}{dpo@fijlkam.it}.
	
	\section{Periodo di conservazione}
	I dati predetti e le altre informazioni costituenti il suo rapporto di tesseramento verranno conservati, anche dopo la cessazione del rapporto, nei limiti di quanto necessario all'espletamento di tutti gli eventuali adempimenti connessi o derivanti dalla natura del rapporto e da obblighi di legge, nonché per eventuali esigenze giudiziarie e di difesa. \\
	I dati relativi alla sua attività sportiva e agonistica, svolta in qualità di soggetto tesserato, potranno essere conservati senza limitazioni temporali per esigenze storico-archivistiche.
	
	\section{Trasferimento al di fuori dell'Unione europea}
	I suoi dati potranno essere oggetto di trasferimento al di fuori dell'Unione europea da parte della Federazione o di responsabili del trattamento di cui la Federazione si avvale per attività connesse alla gestione del rapporto di tesseramento. Tale trasferimento, ove ricorra il caso, sarà legittimato dal consenso specifico dell'interessato, oppure sarà disciplinato con i soggetti importatori mediante il ricorso alle clausole contrattuali standard (SCC) adottate dalla Commissione europea con la Decisione di Esecuzione (UE) 2021/914 ed eventuali successive modifiche. In alternativa, il trasferimento potrà altresì fondarsi su di una decisione di adeguatezza della Commissione, ove presente, oppure sulla base di norme vincolanti d'impresa e/o su di ogni altro strumento consentito dalla normativa applicabile. \\
	Potrà ottenere informazioni sul luogo in cui i suoi dati sono trasferiti e copia di tali dati scrivendo a Federazione Italiana Judo Lotta Karate Arti Marziali, con sede in Via dei Sandolini, 79 – 00122 Ostia Lido (RM), contattabile al seguente indirizzo di posta elettronica: \href{mailto:segreteria.federale@fijlkam.it}{segreteria.federale@fijlkam.it}.
	
	\section*{Responsabile per la protezione dei dati personali (DPO)}
	La Federazione ha nominato un responsabile per la protezione dei dati personali (Data Protection Officer o
	"DPO"), che può essere contattato al seguente indirizzo di posta elettronica: \href{mailto:dpo@fijlkam.it}{dpo@fijlkam.it}.
	
	\section*{Minorenni}
	Ai sensi dell'art. 8 GDPR, qualora l'interessato sia minorenne, il trattamento dei suoi dati personali così come descritto nella presente informativa è condotto con particolari garanzie e precauzioni. Il trattamento in oggetto è pertanto da considerarsi lecito soltanto se e nella misura in cui il conferimento dei dati e la prestazione dei consensi da parte del minore siano autorizzati dai titolari della responsabilità genitoriale.
	
	\begin{center}
		\vspace{0.3in}
		***
		\vspace{0.3in}
	\end{center}
	
	Il sottoscritto, previa identificazione, dichiara di aver ricevuto e preso visione dell'informativa privacy, redatta ai sensi e per gli effetti dell'art. 13 GDPR, sul trattamento dei miei dati personali [\textit{obbligatorio}].
	\begin{center}
		\begin{minipage}{0.4\linewidth}
			\centering
			Data \\[5pt]
			\field{0.6\linewidth}
		\end{minipage}
		\begin{minipage}{0.4\linewidth}
			\centering
			Firma \\[5pt]
			\field{0.6\linewidth}
		\end{minipage}
	\end{center}
	\vspace{0.5in}
	
	Il sottoscritto, previa identificazione, vuole ricevere comunicazioni commerciali da parte della FIJLKAM, nonché per lo svolgimento di attività pubblicitarie e di sponsorizzazione di eventi, tornei e manifestazioni sportive, secondo le modalità descritte al paragrafo 2, lett. C) dell'informativa privacy [\textit{facoltativo}].
	\begin{center}
		\begin{minipage}{0.4\linewidth}
			\centering
			Data \\[5pt]
			\field{0.6\linewidth}
		\end{minipage}
		\begin{minipage}{0.4\linewidth}
			\centering
			Firma \\[5pt]
			\field{0.6\linewidth}
		\end{minipage}
	\end{center}
	\vspace{0.5in}
	
	Il sottoscritto, previa identificazione, presta il consenso alla cessione a terzi (quali, ad esempio, società del settore, sponsor della Federazione) per l'invio di comunicazioni commerciali, secondo le modalità descritte al paragrafo 2, lett. D) dell'informativa privacy [\textit{facoltativo}].
	\begin{center}
		\begin{minipage}{0.4\linewidth}
			\centering
			Data \\[5pt]
			\field{0.6\linewidth}
		\end{minipage}
		\begin{minipage}{0.4\linewidth}
			\centering
			Firma \\[5pt]
			\field{0.6\linewidth}
		\end{minipage}
	\end{center}
	
	\begin{center}
		\vspace{0.3in}
		***
		\vspace{0.3in}
	\end{center}
	
	\begin{minipage}{0.7\linewidth}
		Codice Federale A.S.D.: \underline{\hspace{2cm}06TS2819\hspace{2cm}} \\[10pt]
		Cognome e Nome: \fieldfill
		\begin{center}
			(IN STAMPATELLO)
		\end{center}
		Firma: \fieldfill
	\end{minipage}
	
	\vspace{0.5in}
	[In caso di minori] \\[10pt]
	Firma del genitore (autorizzato anche dall'altro genitore, ove esistente) o dell'esercente la responsabilità genitoriale \\[20pt]
	\field{8cm}
	
	
\end{document}