% !TeX spellcheck = it_IT
% !TeX encoding = UTF-8
\documentclass[
	headerstyle = TitleDojoFijlkam
]{djtsmod}

\title{RICHIESTA DI AMMISSIONE A SOCIO}

\begin{document}
	\maketitle
	
	\recipient{Al Presidente del Consiglio Direttivo \\ A.S.D. DOJO Trieste}
	
	\basedata
	
	\statement{RICHIEDE}
	di essere ammesso come socio della Vs. Associazione, condividendone le finalità e gli scopi. A tal fine, mi impegno fin da ora:
	\begin{itemize}
		\item ad accettare e seguire norme e regole previste dallo Statuto e dal Regolamento Sociale, di cui ho preso visione, o successivamente deliberate dall'Assemblea e/o dal Consiglio Direttivo;
		\item a versare la quota associativa periodica stabilita dal Consiglio Direttivo, nonché le eventuali integrazioni.
	\end{itemize}
	\vspace{0.1in}
	\inlinedatesignfield{Firma}
	
	\vspace{0.3in}
	Il sottoscritto, letta l'informativa ex art 13 del Regolamento UE 2016/679 stampata sul retro della presente, dichiara di essere stato informato sulle finalità e le modalità di trattamento cui sono destinati i dati, i soggetti a cui gli stessi potranno essere comunicati, anche in qualità di incaricati, nonché sul diritto di accesso ai dati personali forniti con facoltà di chiederne l'aggiornamento, la rettifica, l'integrazione e la cancellazione.
	Per quanto sopra esprimo il mio consenso al trattamento dei miei dati personali nelle modalità e per le finalità strettamente connesse e strumentali ai fini statutari dell'associazione.
	\vspace{0.1in}
	\inlinedatesignfield{Firma}
	
	\vspace{0.3in}
	\hrule
	\vspace{0.1in}
	
	\statement{SPAZIO RISERVATO AL C.D.}
	Il Consiglio Direttivo dell'ASD DOJO Trieste HA AMMESSO / NON HA AMMESSO come socio\\ il/la Sig./Sig.ra \fieldfill\; con delibera del \field{2in}. \\
	Motivazione: \fieldfill.
	
	\vspace{0.2in}
	\datesignfield{Firma del Presidente}
	
	\newpage
	
	\tinystatement{INFORMATIVA ex art. 13 del Regolamento (UE) 2016/679}
	\fontsize{7pt}{10pt}\selectfont
	Desideriamo informarLa, in qualità di Titolari del trattamento, che il Regolamento UE/2016/679 General Data Protection Regulation (G.D.P.R.), di immediata applicazione anche in Italia, prevede la tutela delle persone e di altri soggetti rispetto al trattamento dei dati personali.
	L'adesione al sodalizio sportivo e lo svolgimento dell'attività sportiva dilettantistica comporta necessariamente la sua identificazione e il trattamento dei suoi dati personali, pertanto chiediamo la sua approvazione mediante consenso in calce alla presente informativa.
	Ai sensi dell'articolo 13 del G.D.P.R., pertanto, Le forniamo le seguenti informazioni:
	\begin{enumerate}[wide, labelindent = 0pt, noitemsep, topsep = 0pt]
		\item I dati personali [dati anagrafici, recapiti, ecc. (specificare)], da Lei forniti verranno trattati per la partecipazione al sodalizio sportivo e alle attività sportive dilettantistiche organizzate e gestite dallo stesso in conformità agli scopi istituzionali.
		\item Il titolare del trattamento è la ASD DOJO Trieste con sede in via Sinico, 1 – 34139 Trieste, contattabile all'indirizzo mail dojotrieste@libero.it  in persona del Presidente pro tempore.
		\item Finalità e base giuridica del trattamento: i suoi dati personali saranno trattati soltanto nella misura in cui siano indispensabili in relazione alle finalità del sodalizio sportivo, nel rispetto di quanto previsto dalla normativa vigente in materia di protezione dei dati personali, nei limiti stabiliti dalla legge e dai regolamenti e in ogni caso nel rispetto dei principi generali di trasparenza, liceità, correttezza. \\
		Il trattamento dei suoi dati personali viene effettuato  per il perseguimento delle finalità istituzionali dell'ente relative alla gestione e organizzazione di attività sportive dilettantistiche, comprese le attività didattiche e formative nel rispetto di quanto previsto dall'art.90 l.289/02, dal d.lgs. 36/21 e dal d.lgs. 39/21 e comprende ogni utilizzo attinente al rapporto associativo e di tesseramento e alla sua partecipazione alle attività gestite e organizzate dal sodalizio in intestazione, dalle Federazioni Sportive Nazionali e/o Enti di Promozione Sportiva affilianti,  dal Coni, da Sport e Salute s.p.a. o da altri organismi ed enti pubblici e privati promotori di iniziative e progetti in ambito sportivo dilettantistico, esclusivamente in base al suo specifico consenso. \\
		I suoi dati devono necessariamente essere comunicati per il tesseramento alle Federazioni Sportive e/o gli Enti di Promozione Sportiva cui siamo affiliati, al Coni e al Dipartimento dello Sport, tramite Sport e Salute s.p.a., per l'inserimento delle attività sportive dilettantistiche nel registro CONI e nel Registro delle Attività sportive dilettantistiche e potranno essere comunicati ad ogni altro soggetto terzo, pubblico o privato, esclusivamente in relazione ad ogni altro utilizzo inerente il rapporto associativo e di tesseramento e/o connesso alla pratica dell'attività sportiva realizzata per il tramite del sodalizio in intestazione,  in base al suo specifico consenso e non saranno comunicati ad altri soggetti e non saranno oggetto di diffusione.
		\item Il trattamento sarà effettuato con le seguenti modalità: su schede manuali, realizzate anche con l'ausilio di mezzi elettronici, conservate in luoghi chiusi, la cui chiave è detenuta dal Presidente e dagli incaricati dell'amministrazione, ovvero in maniera informatizzata, su un PC posto presso la sede dell'Associazione che è attrezzato adeguatamente contro i rischi informatici (firewall, antivirus, backup periodico dei dati); autorizzati ad accedere a tali dati sono il presidente e gli incaricati dell'amministrazione. Ai sensi dell'art. 4 n. 2 del G.D.P.R, il trattamento dei dati personali potrà consistere nella raccolta, registrazione, organizzazione, consultazione, elaborazione, modificazione, selezione, estrazione, raffronto, utilizzo, interconnessione, blocco, comunicazione, cancellazione e distruzione dei dati. I dati saranno trattati esclusivamente in Italia e nell'ambito dell'Unione Europea
		\item I dati personali saranno conservati per tutto il tempo indispensabile per la corretta tenuta dei libri sociali e/o per procedere alle formalità richieste dalle Federazioni Sportive e/o gli Enti di Promozione Sportiva cui siamo affiliati: tale termine è determinato dal codice civile, dalla normativa fiscale e dalle norme e regolamenti del CONI e delle Federazioni Sportive e/o gli Enti di Promozione Sportiva cui siamo affiliati. La verifica sulla obsolescenza dei dati oggetto di trattamento rispetto alle finalità per le quali sono stati raccolti e trattati viene effettuata periodicamente.
		\item Il conferimento dei dati è obbligatorio per il raggiungimento delle finalità dello statuto dell'Associazione/Società ed è quindi indispensabile per l'accoglimento della sua domanda di ammissione a socio e/o per il tesseramento presso i soggetti indicati al punto precedente; l'eventuale rifiuto a fornirli comporta  l'impossibilità di accogliere la Sua domanda di iscrizione e/o tesseramento, non essendo in tale ipotesi possibile instaurare l'indicato rapporto associativo e/o di tesseramento presso gli enti cui l'Associazione/Società è affiliata.
		\item Il trattamento non riguarderà dati personali rientranti nel novero dei dati c.d. "sensibili", vale a dire i dati personali idonei a rivelare l'origine razziale ed etnica, le convinzioni religiose, filosofiche o di altro genere, le opinioni politiche, l'adesione a partiti, sindacati, associazioni od organizzazioni a carattere religioso, filosofico, politico o sindacale, nonché i dati personali idonei a rivelare lo stato di salute e la vita sessuale, i dati genetici, biometrici, quelli relativi all'orientamento sessuale e giudiziari. La tenuta e conservazione del certificato di idoneità agonistica/non agonistica obbligatoria in base alle vigenti leggi per lo svolgimento della pratica sportiva dilettantistica, non è un dato sanitario ma personale (Garante Privacy nota n. 41878 del 31.12.1998)
		\item In ogni momento Lei potrà esercitare i Suoi diritti di conoscere i dati che La riguardano, sapere come sono stati acquisiti, verificare se sono esatti, completi, aggiornati e ben custoditi, di ricevere i dati in un formato strutturato, di uso comune e leggibile da dispositivo automatico, di revocare il consenso eventualmente prestato relativamente al trattamento dei Suoi dati in qualsiasi momento ed opporsi in tutto od in parte, all'utilizzo degli stessi come sanciti dagli artt. da 15 a 22 del G.D.P.R. Tali diritti possono essere esercitati attraverso specifica istanza da indirizzare tramite raccomandata – o PEC - al Titolare del trattamento indicata al punto 2.
		\item Lei ha in diritto di revocare il consenso in qualsiasi momento senza pregiudicare la liceità del trattamento basata sul consenso prestato prima della revoca. Tale diritto potrà essere esercitato inviando la revoca del consenso all'indirizzo e-mail indicato nel precedente punto 2. Resta salvo il diritto di proporre reclamo al Garante per la protezione dei dati personali (www.garanteprivacy.it)
		\item Non esiste alcun processo decisionale automatizzato, né alcuna attività di profilazione di cui all'articolo 22, paragrafi 1 e 4 del G.D.P.R.
	\end{enumerate}

	\fontsize{9pt}{10pt}\selectfont
	
	\tinystatement{DICHIARAZIONE DI CONSENSO AL TRATTAMENTO DEI DATI PERSONALI}
	Il/la sottoscritto/a \fieldfill\; nato/a a \field{1.5in} il \field{1.5in}, \\ residente a \field{1.5in} via \fieldfill, C.F. \field{2in}, \\
	letta l'informatica che precede.
	
	\tinystatement{ESPRIMO IL CONSENSO}
	\squarelabel\; SI \quad \squarelabel\; NO \quad al trattamento dei miei dati personali nelle modalità e per le finalità indicate al punto 3, strettamente connesse e strumentali alla gestione del rapporto associativo e di tesseramento per la pratica dell'attività sportiva dilettantistica.
	\vspace{0.05in}
	\inlinedatesignfield{Firma}
	
	\tinystatement{ESPRIMO IL CONSENSO ALL'UTILIZZO DI IMMAGINI FOTOGRAFICHE e/o AUDIOVISIVE}
	\squarelabel\; SI \quad \squarelabel\; NO \quad all'utilizzo e alla pubblicazione in forma cartacea o elettronica, a titolo gratuito, delle immagini fotografiche o immagini audiovisive a lei riferite o di dati a lei riferiti, per il perseguimento dei fini istituzionali del sodalizio nonché per attività di informazione, divulgazione e promozione, in ogni caso sempre nel rispetto dei diritti e delle libertà fondamentali dell'interessato e coerentemente con le politiche e le finalità del Titolare. Autorizzo l'associazione ad utilizzarle e diffonderle liberamente con qualsiasi mezzo, anche multimediale, in conformità agli scopi istituzionali e a scopo divulgativo con finalità di promozione.
	\vspace{0.05in}
	\inlinedatesignfield{Firma}
	
\end{document}