% !TeX spellcheck = it_IT
% !TeX encoding = UTF-8
\documentclass{djtsasddoc}

\title{MODELLO ORGANIZZATIVO E DI CONTROLLO DELL'ATTIVITÀ SPORTIVA \\[10pt] {\normalfont\large per la tutela dei minori e per la prevenzione delle molestie, della violenza di genere e di ogni altra condizione di discriminazione, c.d. politiche di Safeguarding dell'Associazione Sportiva Dilettantistica DOJO Trieste}}
\date{Trieste, 22 agosto 2024}

\usepackage{titlesec}
\titleformat{\section}{\normalfont\large\bfseries}{Art. \thesection\ -\ }{0pt}{}
\titlespacing*{\section}{0pt}{0.4in}{0.2in}

\begin{document}
	\maketitle
	
	\section*{Premessa}
	L'Associazione Sportiva Dilettantistica DOJO Trieste, i suoi Tesserati e le sue Tesserate, nelle rispettive qualità di Affiliata e Tesserati FIJLKAM, si conformano, unitamente alla Federazione stessa, alle disposizioni di cui ai D. Lgs. n. 36 e 39 del 28 febbraio 2021, nonché alle disposizioni emanate dalla Giunta Nazionale del CONI e dall'Osservatorio permanente del CONI per le Politiche di \textit{Safeguarding} in materia, nonché al Regolamento per la tutela dei Tesserati - \textit{Safeguarding Policy}, approvato dal Consiglio Federale FIJLKAM in data 1 dicembre 2023 e alle Linee Guida Federali pubblicate il 31 agosto 2023, in base alle quali viene emanato il seguente Modello Organizzativo e di Controllo dell'Attività Sportiva  a tutela dei minori  e per la prevenzione delle molestie, della violenza di genere e di ogni altra condizione di discriminazione.
	Il Modello Organizzativo e di Controllo dell'Attività Sportiva viene adottato con delibera di data 22 agosto 2024 del Consiglio Direttivo secondo la procedura prevista dallo Statuto della dell'ASD DOJO Trieste.
	
	\section{Finalità}
	\begin{enumerate}
		\item Il presente documento regolamenta e disciplina gli strumenti per la prevenzione e il contrasto di ogni forma di abuso, molestia, violenza di genere o discriminazione per ragioni di etnia, religione, convinzioni personali, disabilità, età o orientamento sessuale ovvero per le ragioni di cui al D.lgs. n.198/2006 attuati in danno dei Tesserati, specie se minori d'età, nell'ambito dell'Associazione Sportiva Dilettantistica DOJO Trieste, di seguito per brevità anche solo Associazione o ASD.
		\item Le norme e le previsioni contenute nel presente regolamento richiamano e sono conformi alle Linee Guida adottate dalla FIJLKAM attualmente in vigore e costituiscono l'insieme delle regole di condotta a cui tutti gli appartenenti della ASD sono tenuti ad uniformarsi al fine di:
		\begin{enumerate}
			\item promuovere il diritto di tutti i tesserati ad essere tutelati da ogni forma di abuso, violenza discriminazione;
			\item promuovere una cultura e un ambiente inclusivi che assicurino la dignità e il rispetto dei diritti di tutti i Tesserati, specie se minori, e garantiscano l'uguaglianza e l'equità, nonché valorizzino le diversità;
			\item rendere consapevoli i Tesserati in ordine ai propri e altrui diritti, doveri, obblighi e responsabilità;
			\item individuare e attuare adeguate misure, procedure e politiche di \textit{Safeguarding}, anche in conformità alle raccomandazioni del \textit{Safeguarding Office} istituito dalla FIJLKAM volte a ridurre i rischi di condotte lesive dei diritti, specie nei confronti di Tesserati minori;
			\item provvedere alla gestione tempestiva, efficace e riservata delle segnalazioni di fenomeni di abuso, violenza e discriminazione e alla tutela dei segnalanti;
			\item informare i Tesserati, anche minori, sulle misure e procedure di prevenzione e contrasto ai fenomeni di abuso, violenza e discriminazione e, in particolar modo, sulle procedure per la segnalazione degli stessi;
			\item incentivare la partecipazione dei componenti del sodalizio alle iniziative organizzate dalla FIJLKAM nell'ambito delle politiche di \textit{Safeguarding};
			\item garantire il coinvolgimento di tutti coloro che partecipano con qualsiasi funzione o titolo all'attività sportiva nell'attuazione delle misure, procedure e politiche di \textit{Safeguarding} della ASD.
		\end{enumerate}
	\end{enumerate}
	
	\section{Ambito di applicazione}
	\begin{enumerate}
		\item I soggetti tenuti al rispetto del presente documento sono:
		\begin{enumerate}
			\item tutti i tesserati della ASD DOJO Trieste;
			\item tutti coloro che intrattengono rapporti di lavoro o volontariato con la ASD;
			\item tutti coloro che, a qualsiasi titolo, intrattengono rapporti con la ASD.
		\end{enumerate}
	\end{enumerate}
	
	\section{Norme di condotta}
	\begin{enumerate}
		\item È onere della ASD strutturarsi in modo tale da dare attuazione alle finalità indicate all'art. 1. uniformando i propri comportamenti alle norme di condotta di seguito indicate:
		\begin{enumerate}
			\item \textbf{assicurare un ambiente ispirato a principi di uguaglianza e di tutela della libertà, della dignità e dell'inviolabilità della persona} predisponendo turni di allenamento e la partecipazione alle gare evitando discriminazioni tra gli atleti in base sesso, all'etnia, appartenenza culturale ecc. e prevedendo, in presenza di minori appartenenti a categorie svantaggiate la loro equa suddivisione in squadre o gruppi di allenamento in modo da facilitare l'integrazione. In particolare, l'Associazione che in futuro si strutturerà per accogliere tutte le categorie federali, al momento accoglie Tesserati dalla Categoria Bambini/e “B” in su (dai 6 anni a seguire). I turni di allenamento sono e saranno organizzati esclusivamente in base all'età e ai colori delle cinture (bianca e gialla, colorate, ecc). Gli orari di allenamento sono e saranno necessariamente diversificati in base all'età, iniziando dai bambini per finire la sera con i masters, senza che ciò possa costituire discriminazione. La partecipazione alle gare verrà concordata (come già avviene) con gli Atleti e/o esercenti la potestà genitoriale per i minori, in base all'impegno profuso negli allenamenti, al grado di preparazione, alle condizioni fisiche e psicologiche e anche in base agli impegni scolastici, familiari e/o lavorativi per gli adulti. 
			\item \textbf{riservare ad ogni Tesserato attenzione, impegno e rispetto, senza distinzioni di età, etnia, condizione sociale, opinione politica, convinzione religiosa, genere, orientamento sessuale, disabilità e altro} tramite la presenza di un numero adeguato di tecnici in relazione alla composizione di ciascun gruppo di atleti consentendo così a tutti di integrarsi nel gruppo di allenamento.
			\item \textbf{far svolgere l'attività sportiva nel rispetto dello sviluppo fisico, sportivo ed emotivo dell'allievo, tenendo in considerazione anche interessi e bisogni dello stesso} attraverso l'ascolto dei minori al fine di comprendere quali siano le loro ambizioni e i loro desideri in ambito sportivo, continuando l'ascolto degli stessi sino al raggiungimento della maggiore età, e anche in seguito, al fine di individuare e comprendere eventuali mutate ambizioni e diversi desideri, onde poter programmare per ciascun atleta l'attività sportiva o la partecipazione alle varie tipologie di competizioni in modo da tener conto delle capacità individuali e delle aspirazioni di ciascuno;
			\item \textbf{prestare la dovuta attenzione ad eventuali situazioni di disagio anche derivante da disturbi dell'alimentazione, percepiti o conosciuti anche indirettamente, con particolare attenzione a circostanze che riguardino i minori} affiancando ai tecnici, se del caso, delle figure professionali specializzate e/o prevedendo durante gli allenamenti la presenza di figure ulteriori rispetto al tecnico che possano monitorare il comportamento degli atleti e organizzando percorsi volti a favorire l'educazione alimentare. Le figure professionali specializzate saranno, se necessario, individuate tra quelle operanti nella rete per il trattamento dei disturbi del comportamento alimentare (DCA) del Sistema Sociale e Sanitario della Regione Autonoma Friuli Venezia Giulia consistente in una rete di servizi articolata a livello territoriale che fornisce una risposta a questo gruppo di disturbi definendo percorsi diagnostico terapeutico riabilitativi personalizzati\footnote{\href{https://www.regione.fvg.it/rafvg/cms/RAFVG/salute-sociale/sistema-sociale-sanitario/FOGLIA29/}{https://www.regione.fvg.it/rafvg/cms/RAFVG/salute-sociale/sistema-sociale-sanitario/FOGLIA29}}. A Trieste, in particolare per i minori, è operante il Centro presso l'IRCCS materno infantile \textit{Burlo Garofolo}. Sarà inoltre individuato tra i dirigenti e/o istruttori e/o volontari una figura di riferimento che, in relazione all'età degli atleti, possa dialogare con loro al fine di scorgere segni di malessere. Saranno in ogni caso vietate pratiche “estreme” di dimagrimento pre-gara consistenti in digiuni, sudorazione forzata, ecc.
			\item \textbf{segnalare, senza indugio, ogni circostanza di interesse agli esercenti la responsabilità genitoriale o tutoria ovvero ai soggetti preposti alla vigilanza}, individuando nel Presidente della ASD o in un dirigente e/o istruttore, scelto anche in base al rapporto fiduciario con gli atleti, sia minorenni che maggiorenni interessati, il soggetto che deve provvedere alla segnalazione. Sarà prevista dal Regolamento Sociale la segnalazione agli esercenti la responsabilità genitoriale o tutori ovvero ai soggetti preposti alla vigilanza, l'assenza reiterata da gare o allenamenti compiute dai minori che hanno raggiunto l'età per muoversi in autonomia e che non vengono accompagnati alle gare in città e agli allenamenti presso la palestra. Saranno inoltre segnalate ai predetti soggetti situazioni di interesse di natura sportiva, quali comportamenti anomali nel corso degli allenamenti, improvvisa svogliatezza nella pratica della disciplina, stanchezza, irascibilità, nonché di natura extra sportiva quali la rappresentazione da parte dell'atleta di situazioni di disagio in ambiente familiare, scolastico, nella sfera delle amicizie e in campo sentimentale. Da ultimo saranno ovviamente segnalate anche situazioni di interesse sanitario palesatesi prima, durante e dopo gli allenamenti.
			\item \textbf{confrontarsi con il Responsabile delle Politiche di \textit{Safeguarding} nominato dalla ASD ove si abbia il sospetto circa il compimento di condotte rilevanti ai sensi del presente documento};
			\item \textbf{attuare idonee iniziative volte al contrasto dei fenomeni di abuso, violenza e discriminazione adottando i seguenti comportamenti}:
			\begin{itemize}
				\item sollecitare atleti, tecnici e dirigenti all'uso di un linguaggio appropriato e comunque evitare l'uso di espressioni discriminatorie, sessiste, o di matrice razzista;
				\item richiedere ai tecnici e dirigenti di instaurare tra loro rapporti professionali evitando situazioni di imbarazzo considerando che alla struttura, di proprietà privata, dove di svolge l'attività, si accede da un corridoio sul quale si affacciano sia i due spogliatoi, uno maschile e uno femminile, che la palestra, dotata di una piccola tribuna per il pubblico e di un ampio magazzino sul retro al quale si accede dalla palestra stessa. In particolare, 
				\item i tecnici si cambieranno negli spogliatoi, rispettivamente in quello maschile gli uomini e in quello femminile le donne, prima dell'inizio e dopo la fine delle lezioni;
				\item gli accompagnatori dei minori sosteranno nei corridoi senza entrare negli spogliatoi. Soltanto per i preagonisti delle categorie Bambini/e “A” (4 e 5 anni) , Bambini/e “B” (6 e 7 anni), Fanciulli/e (8 e 9 anni) e Ragazzi/e (10 e 11) anni sarà permesso l'ingresso negli spogliatoi di un massimo di due esercenti la responsabilità genitoriale o tutoria ovvero preposti alla vigilanza, scelti fra tutti gli appartenenti a dette categorie tramite patto di corresponsabilità concordato con l'ASD. I soggetti designati, anche con turnazione, dovranno essere preferibilmente dello stesso sesso dei minori. I soggetti designati potranno sostare negli spogliatoi solo per il tempo necessario del cambio e non potranno rimanere negli spogliatoi durante le lezioni;
				\item in caso di soggetti minorenni o maggiorenni in transito con riferimento all'identità di genere il presente Modello dovrà essere aggiornato in considerazione degli spazi come descritti;
				\item gli accompagnatori dei minori come sopra designati potranno assistere alle lezioni una volta al mese sostando nella tribuna, in assoluto silenzio, senza interferire in alcun modo con l'attività sportiva e l'operato degli istruttori;
				\item qualora ve ne fosse la necessità, nel corso della lezione un istruttore accompagnerà gli atleti sino agli 11 anni all'ingresso degli spogliatoi, rimanendone fuori degli stessi senza chiudere la porta, per permettere all'atleta di andare in bagno (separato dallo spogliatoio da due porte) da solo e in sicurezza;
				\item la doccia può essere effettuata solo dagli atleti maggiorenni in un locale dedicato adiacente agli spogliatoi, con la necessaria turnazione tra atleti maschi e atlete femmine. Gli istruttori potranno fare la doccia per ultimi, dopo gli atleti;
				\item all'interno degli spogliatoi gli atleti dovranno mantenere una condotta educata come quella che devono tenere sul tatami, nel rispetto reciproco. Non potranno essere scattate foto degli altri atleti semisvestiti e non vestiti e non potranno essere effettuate riprese in tali condizioni. Non potranno essere effettuati video “in diretta” dagli spogliatoi per non ledere la privacy altrui. Non potranno essere rivolte ingiurie nei confronti degli altri atleti né tantomeno potranno essere intonati cori denigratori. Non potranno essere provate e/o ripetute azioni di allenamento negli spogliatoi. Non potranno essere creati gruppi social al solo fine di denigrare un altro atleta;
				\item i dirigenti, gli istruttori e i volontari non potranno accompagnare a casa atleti minori a meno che non siano legati a essi da vincolo di parentela quali esercenti la responsabilità genitoriale o tutoria ovvero preposti alla vigilanza, ad eccezione dei casi in cui ne vengano espressamente richiesti per iscritto dagli esercenti la responsabilità genitoriale o tutoria ovvero preposti alla vigilanza;
				\item in caso di trasferte, individuato l'elenco completo degli atleti partecipanti, si individuerà lo staff di accompagnatori interno all'ASD (es. dirigenti, istruttori, preparatori, collaboratori anche volontari, ecc.) e si reperirà un mezzo di trasporto idoneo sotto ogni punto di vista a contenere atleti e staff. In caso di viaggio con singole automobili e di atleti minori i componenti dello staff presenti nella stessa auto dovranno essere almeno due e dovranno ricevere l'autorizzazione scritta da parte degli esercenti la responsabilità genitoriale o tutoria ovvero preposti alla vigilanza; questi ultimi in casi particolari, potranno autorizzare anche un singolo accompagnatore. I minori saranno "prelevati" e "riportati" in un punto di prelievo individuato nel piazzale antistante la palestra o altro da individuarsi di volta in volta. Sarà richiesto alla società ospitante di garantire la presenza di spogliatoi idonei per gli atleti, con suddivisione dei locali per sesso anche in considerazione dell'eventuale identità di genere e che possano essere utilizzati agevolmente dagli atleti che presentano disabilità. Per i pasti verranno individuati locali idonei a ospitare in unica tavolata o in tavoli limitrofi atleti e staff. In caso di trasferta con pernottamento, in base al numero degli accompagnatori, degli atleti e delle camere verrà creato un elenco, il c.d. roaming list. Gli atleti  minorenni e maggiorenni non dovranno dormire nelle stesse camere degli accompagnatori fatta eccezione per coloro che sono legati da vincolo di parentela. In caso di pernotto fuori sede di atleti minori, l'accesso alla stanza da questi occupata sarà limitato a tecnici e dirigenti esclusivamente per finalità di controllo. Gli esercenti la responsabilità genitoriale o tutoria ovvero i preposti alla vigilanza dovranno conoscere in anticipo  i dettagli dell'organizzazione della trasferta e la composizione delle stanze. Durante la permanenza degli atleti negli spogliatoi e negli alloggi della trasferta saranno valide le stesse regole di comportamento vigenti per lo spogliatoio della palestra.
			\end{itemize}
			\item \textbf{prevenire, durante gli allenamenti e in gara, tutti i comportamenti e le condotte sopra descritti con azioni di sensibilizzazione e controllo} quali l'organizzazione di riunioni periodiche che coinvolgano i tecnici e i dirigenti e gli esercenti la responsabilità genitoriale o tutoria ovvero i preposti alla vigilanza nel cui ambito illustrare le politiche di salvaguardia dei minori e le azioni che si intendono intraprendere e nelle quali discutere delle criticità emerse nel corso della stagione sportiva 
			\item \textbf{spiegare in modo chiaro a coloro che assistono allo svolgimento di allenamenti, gare o manifestazioni sportive, di astenersi da apprezzamenti, commenti e valutazioni che non siano strettamente inerenti alla prestazione sportiva in quanto potrebbero essere lesivi della dignità, del decoro e della sensibilità della persona} attraverso l'illustrazione del principio di responsabilità oggettiva in capo all'ASD che potrebbe comportare anche l'irrogazione di sanzioni in capo alla ASD dai Giudici Sportivi Territoriali e/o Federali, conseguenti l'irrogazione di provvedimenti sanzionatori a carico di coloro che durante le gare tengano un comportamento non adeguato. Per coloro che assistono agli allenamenti si è già scritto al punto g).
			\item \textbf{favorire la rappresentanza paritaria di genere, nel rispetto della normativa applicabile};
			\item \textbf{rendere consapevoli i tesserati in ordine ai propri diritti, doveri, obblighi e responsabilità adottando le seguenti misure}:
			\begin{itemize}
				\item affissione presso la sede operativa dell'Affiliata del modello organizzativo e del Codice di condotta adottato e degli eventuali aggiornamenti, integrazioni o modifiche e/o pubblicazione dello  stesso sulla homepage del sito della ASD;
				\item affissione presso la sede della ASD e/o pubblicazione sulla homepage del sito della ASD del nominativo del \textit{Safeguarding} nominato dal sodalizio con indicazione dell'indirizzo e-mail per poterlo contattare;
				\item comunicazione, anche alternativamente a mezzo mail e/o PEC al momento del tesseramento, agli atleti o agli esercenti la responsabilità genitoriale o tutoria ovvero preposti alla vigilanza, se minorenni, del modello organizzativo e codice di condotta adottato dalla ASD, nonché comunicazione del nominativo del Responsabile del \textit{Safeguarding} nominato dalla ASD e delle le procedure da seguire per la segnalazione di comportamenti lesivi al detto Responsabile tramite la mail dedicata; 
				\item informazione ai tesserati e agli esercenti la responsabilità genitoriale o tutoria ovvero preposti alla vigilanza, se minorenni, circa le misure adottate dalla ASD per la prevenzione e il contrasto a comportamenti lesivi consistenti anche nell'organizzazione nel corso della stagione sportiva di incontri e seminari  con il Responsabile del \textit{Safeguarding} ed esperti del settore (psicologi, medici nutrizionisti, medici sportivi, ecc.) con cui discutere della tematica anche a fine di pervenire a soluzioni condivise.
				\item Evidenziare ai tesserati e agli esercenti la responsabilità genitoriale o tutoria ovvero preposti alla vigilanza la necessità di informare i medici di medicina generale e/o i pediatri in merito allo svolgimento dell'attività sportiva agonistica in modo da evitare l'assunzione di farmaci dopanti.
			\end{itemize}
		\end{enumerate}
	\end{enumerate}
	
	\section{Tutela dei minori: Obblighi}
	\begin{enumerate}
		\item Tutti coloro che in ambito associativo - a prescindere dalla forma del rapporto instaurato - svolgano funzioni che comportano contatti diretti e regolari con minori devono fornire semestralmente la copia del certificato del casellario giudiziale ai sensi della normativa vigente. Solo nelle more fra la richiesta e l'ottenimento del primo certificato può essere fornita un'autocertificazione, fatta salva la facoltà per l'ASD di subordinare e rinviare i contatti diretti e regolari con i minori alla presentazione del certificato.
	\end{enumerate}
	
	\section{Responsabile delle politiche di salvaguardia nominato dalla ASD}
	\begin{enumerate}
		\item Allo scopo di prevenire e contrastare ogni tipo di abuso, violenza e discriminazione sui Tesserati nonché per garantire la protezione dell'integrità fisica e morale degli sportivi, anche ai sensi dell'art. 33, comma 6, del D.lgs. n. 36/2021, la ASD nomina un Responsabile contro abusi, violenze e discriminazioni e lo comunica alla FIJLKAM all'atto di affiliazione o come e nei termini diversamenti indicati per legge.
		\item Il Responsabile contro abusi, violenze e discriminazioni dovrà essere prescelto tra i tesserati di comprovata moralità e competenza in possesso dei seguenti requisiti:
		\begin{enumerate}
			\item qualora non tesserato deve essere regolarmente tesserato alla FIJLKAM, salvo diverse future disposizioni;
			\item non aver riportato condanne penali anche non passate in giudicato per i seguenti reati: art 600-bis(prostituzione minorile); 600-ter (pornografia minorile), 600-quater (detenzione o accesso a materiale pornografico), 600-quater.1 (pornografia virtuale), 600-quinquies (iniziative turistiche volte allosfruttamento della prostituzione minorile), 604-bis (propaganda e istigazione a delinquere per motivi discriminazione etnica e religiosa), 604-ter, (circostanze aggravanti) 609-bis (violenza sessuale), 609-ter (circostanze aggravanti), 609-quater (atti sessuali con minorenne), 609-quinquies (corruzionedi minorenne), 609-octies (violenza sessuale di gruppo), 609-undecies (adescamento di minorenni).c non aver riportato nell'ultimo decennio, salva riabilitazione, squalifiche o inibizioni sportive definitive complessivamente superiori ad un anno, da parte delle FSN, delle DSA, degli EPS e del CONI o di organismi sportivi internazionali riconosciuti;
			\item aver seguito i corsi di aggiornamento previsti dalla FIJLKAM e/o essere in possesso dei titoli abilitativi eventualmente previsti dai regolamenti federali.
		\end{enumerate}
		\item La nomina del Responsabile è adeguatamente resa pubblica mediante immediata affissione presso la sede e pubblicazione sulla rispettiva \textit{homepage} del sito internet della ASD e inserita nel sistema gestionale federale, secondo le procedure previste dalla regolamentazione federale.
		\item Il Responsabile dura in carica un anno e può essere riconfermato.
		\item In caso di cessazione del ruolo di Responsabile contro abusi, violenze e discriminazioni, per dimissioni o per altro motivo, il sodalizio provvede entro 30 (trenta) giorni alla nomina di un nuovo Responsabile inserendola nel sistema gestionale federale, secondo le procedure previste dalla regolamentazione federale.
		\item La nomina di Responsabile contro abusi, violenze e discriminazioni può essere revocata ancora prima della scadenza del termine per gravi irregolarità di gestione o di funzionamento, ovvero per il venir meno dei requisiti necessari alla sua nomina, con provvedimento motivato dell'organo preposto del sodalizio. Della revoca e delle motivazioni è data tempestiva notizia al \textit{Safeguarding Office} della FIJLKAM. Il sodalizio provvede alla sostituzione con le modalità di cui al precedente comma.
		\item Il Responsabile è tenuto a:
		\begin{enumerate}
			\item promuovere la corretta applicazione del Regolamento per la prevenzione e il contrasto ad abusi,violenze e discriminazioni sui Tesserati della FIJLKAM nell'ambito della ASD, nonché l'osservanza e l'aggiornamento dei Modelli organizzativi e di controllo dell'attività sportiva e dei Codici di condotta adottati dagli stessi;
			\item adottare le opportune iniziative, anche con carattere d'urgenza, per prevenire e contrastare nell'ambito del proprio sodalizio ogni forma di abuso, violenza e discriminazione nonché ogni iniziativa di sensibilizzazione che ritiene utile e opportuna;
			\item segnalare al \textit{Safeguarding Office} della FIJLKAM eventuali condotte rilevanti e fornire allo stesso ogni informazione o documentazione richiesta;
			\item rispettare gli obblighi di riservatezza imposti dai Regolamenti FIJLKAM;
			\item formulare all'organo preposto le proposte di aggiornamento dei Modelli organizzativi e di controllo sull'attività sportiva e dei Codici di condotta, tenendo conto delle caratteristiche del sodalizio;
			\item valutare annualmente l'adeguatezza dei modelli organizzativi e di controllo dell'attività sportiva e dei codici di condotta nell'ambito del proprio sodalizio, eventualmente sviluppando e attuando sulla base di tale valutazione un piano d'azione al fine risolvere le criticità riscontrate; 
			\item partecipare all'attività obbligatoria formativa organizzata dalla FIJLKAM.
		\end{enumerate}
	\end{enumerate}
	
	\section{Dovere di segnalazione}
	\begin{enumerate}
		\item Chiunque venga a conoscenza di comportamenti rilevanti come individuati dal Regolamento e dalle linee guida predisposte dalla FIJLKAM e nel presente documento integralmente richiamate, è tenuto a darne immediata comunicazione al Responsabile \textit{Safeguarding} nominato dalla ASD.
		\item Chiunque sospetta comportamenti rilevanti ai sensi del presente Regolamento può confrontarsi con il Responsabile \textit{Safeguarding} nominato dalla ASD.
	\end{enumerate}
	
	\section{Diffusione ed attuazione}
	\begin{enumerate}
		\item Il presente documento è pubblicato sul sito internet del sodalizio, e/o affisso presso la sede operativa dello stesso ed è portato a conoscenza di tutti i collaboratori, qualunque sia il motivo della collaborazione, al momento in cui si instaura il rapporto con la ASD.
	\end{enumerate}
	
	\section{Sanzioni}
	\begin{enumerate}
		\item Restando impregiudicata l'applicazione delle sanzioni previste dai Regolamenti FIJLKAM, a carico di tutti coloro che sono assoggettati, ai sensi delle previsioni di cui all'art.2, tra le categorie tenute all'osservanza delle disposizioni contenute nel presente documento e che pongano in essere comportamenti contrari a quanto ivi indicato, potranno essere irrogate sanzioni da modulare in base alla gravità del comportamento tenuto, qualora previsto dal rapporto contrattuale instaurato con il tesserato ovvero dalle norme statutarie e regolamentari della ASD.
	\end{enumerate}
	
	\section{Norme finali}
	\begin{enumerate}
		\item Il presente documento è aggiornato dall'organo direttivo della ASD con cadenza almeno quadriennale e ogni qual volta necessario al fine di recepire le eventuali ulteriori disposizioni emanate dalla Giunta Nazionale del CONI, eventuali modifiche e integrazioni dei Principi Fondamentali approvati dall'Osservatorio Permanente del CONI per le politiche di \textit{Safeguarding} ovvero le sue raccomandazioni nonché eventuali modifiche e integrazioni delle disposizioni della FIJLKAM.
		\item Eventuali proposte di modifiche al presente documento dovranno essere sottoposte ed approvate dall'organo preposto della ASD.
		\item Per quanto non esplicitamente previsto si rimanda a quanto prescritto dallo Statuto della FIJLKAM, nonché nel Regolamento per la prevenzione e il contrasto ad abusi, violenze e discriminazioni sui Tesserati e nel Codice di Condotta della ASD.
		\item Il presente Regolamento, approvato dal Consiglio Direttivo nella seduta del 22.08.2024, entra in vigore il giorno successivo alla sua pubblicazione sul sito internet della ASD.
	\end{enumerate}
	
\end{document}