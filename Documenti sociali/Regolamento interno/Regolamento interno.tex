% !TeX encoding = UTF-8
% !TeX spellcheck = it_IT
\documentclass{djtsdoc}

\title{REGOLAMENTO INTERNO \\[2pt] dell'ASD DOJO Trieste}
\date{Approvato con Delibera del Consiglio Direttivo del 25 novembre 2024 \\ Approvato con Delibera Assembleare del 27 novembre 2024}

\usepackage{titlesec}
\titleformat{\section}{\normalfont\large\bfseries}{Art. \thesection\ -\ }{0pt}{}
\titlespacing*{\section}{0pt}{0.4in}{0.2in}

\begin{document}
	\maketitle
	
	\section*{Premessa}
	L'Associazione Sportiva Dilettantistica DOJO Trieste, Affiliata FIJLKAM, al fine di uniformarsi alle disposizioni di cui ai D. Lgs. n. 36 e 39 del 28 febbraio 2021, nonché alle disposizioni emanate dalla Giunta Nazionale e dall'Osservatorio permanente per le Politiche di Safeguarding del CONI ed anche al Regolamento per la tutela dei Tesserati - Safeguarding Policy, approvato dal Consiglio Federale FIJLKAM in data 1 dicembre 2023 e alle Linee Guida Federali pubblicate il 31 agosto 2023 in materia di Safeguarding,
	\begin{enumerate}[label = \alph*.]
		\item in data 24.11.2023 ha approvato il nuovo Statuto Registrato in data 08.01.2024;
		\item in data 28.06.2024 ha nominato il Responsabile Safeguarding;
		\item in data 22.08.2024 ha adottato il Codice di Condotta;
		\item in data 22.08.2024 ha adottato il Modello Organizzativo e di Controllo dell'Attività Sportiva a tutela dei minori e per la prevenzione delle molestie, della violenza di genere e di ogni altra condizione di discriminazione, c.d. "MOG".
	\end{enumerate}
	Successivamente, al fine di adeguare e uniformare fra loro i documenti associativi, è stato redatto ai sensi e per gli effetti dell'art. 17, comma 1, lettere k) e l) dello Statuto e delle indicazioni contenute nel Codice di Condotta e nel MOG, il presente regolamento.
	
	\section{Principi generali}
	\begin{enumerate}
		\item Chi si dedica allo studio del Judo, deve pensare che le ore di allenamento nel Dojo non servono unicamente a fare di lui un atleta, ma servono soprattutto alla sua crescita.
		\item Il Judoka con la pratica del Judo imparerà le virtù fondamentali per una persona completa: volontà, modestia, sincerità, coraggio, capacità di sopportare, tenacia e imparerà ad essere di esempio agli altri in ogni situazione. Il compito degli istruttori sarà quello di trasmettere l’essenza del Judo e l’importanza del rispetto reciproco.
		\item Nell'animo di un buon Judoka deve esserci sempre un senso di amicizia sincera verso i suoi compagni di palestra ed anche verso tutti i Judoka e le persone che gli capiterà di frequentare. L'amicizia e la collaborazione costruiranno le basi per far prosperare una comunità.
	\end{enumerate}
	
	\section{Soggetti tenuti all'osservanza del Regolamento}
	I soci, i tesserati, coloro che intrattengono rapporti di lavoro o di volontariato con l'ASD nonché tutti coloro che, a qualsiasi titolo, intrattengono rapporti con la ASD, sono tenuti a conoscere e rispettare il presente Regolamento.
	
	\section{Soggetti tenuti al tesseramento e obbligo di associarsi all'ASD. Termine per il pagamento e per il rinnovo della quota associativa}
	\begin{enumerate}
		\item Coloro i quali intendono svolgere attività sportiva amatoriale o agonistica con la ASD devono tesserarsi alla FIJLKAM (o alle Federazioni Sportive Nazionali o Discipline Sportive Associate o Enti di Promozione Sportiva) ai quali l'Associazione sarà affiliata.
		\item Al fine di incrementare la partecipazione dei Tesserati alla vita associativa, questi sono tenuti anche ad acquisire la qualifica di socio dell'ASD con le modalità indicate nell'art. 4 dello Statuto.
		\item I nuovi tesserati pagheranno la quota di tesseramento al momento dell'iscrizione al corso e la quota associativa non appena sarà loro comunicata la loro ammissione a socio ex art. 4 Statuto. 
		\item Il nuovo tesseramento, effettuato dal 01 gennaio al 31 agosto scade il 31 dicembre dello stesso anno ad eccezione di coloro che si tesserano dal 01 settembre per cui il tesseramento ha effetto fino al 31 dicembre dell'anno successivo, come da regolamento federale.
		\item Il termine per tutti i rinnovi della quota associativa scade il 31 marzo di ogni anno.
	\end{enumerate}
	
	\section{Tesseramento}
	\begin{enumerate}
		\item Ai sensi dell'art. 16, comma 2, del d.lgs. 36/2021 , così come modificato dal d. lgs. n. 163 del 5 ottobre 2022) il minore che abbia compiuto 14 anni deve controfirmare il modulo di tesseramento per lui compilato dall'esercente la responsabilità genitoriale o tutoria. 
		\item Il socio tesserato che intende rinnovare il tesseramento o che non intende rinnovare il tesseramento dovrà comunicare la sua decisione all'ASD con PEC o lettera raccomandata a.r. o mail semplice entro il 31 dicembre, come indicato nelle "Norme Affiliazioni e Tesseramenti FIJLKAM". Il mancato rinnovo del tesseramento comporterà per il socio l'impossibilità di praticare attività sportiva agonistica e non. 
		\item In mancanza di disdetta da parte del socio, la quota del tesseramento dovrà essere accreditata sul conto corrente dell'ASD  tassativamente entro e non oltre il 31 dicembre.
		\item Il tesserato deve produrre la certificazione medica richiesta dalla vigente legislazione nonché dalle "Norme Affiliazioni e Tesseramenti FIJLKAM". In particolare, salvo modifiche legislative o federali, la certificazione deve essere rilasciata da medici sportivi in caso di idoneità agonistica, e dai medici di medicina generale per l'attività non agonistica. Entrambi i certificati devono essere consegnati alla ASD in originale entro la data di scadenza del precedente certificato.
		\item Il certificato medico è obbligatorio anche per i minori di 6 anni nei casi specifici indicati dal pediatra, in ottemperanza al Decreto del Ministro della Salute di concerto con il Ministro dello Sport del 28.02.2018.
	\end{enumerate}
	
	\section{Determinazione e pagamento quote per la partecipazione ai corsi}
	\begin{enumerate}
		\item ll Consiglio Direttivo stabilisce annualmente l'ammontare delle quote per la partecipazione ai corsi; in caso di iscrizione a corso già iniziato dette quote verranno ridotte proporzionalmente alla durata del corso stesso. 
		\item Il Consiglio Direttivo potrà inoltre deliberare in merito alla modulazione dell'ammontare  delle quote per la partecipazione ai corsi tenendo conto delle condizioni personali e familiari del tesserato.
	\end{enumerate}
	
	\section{Nomine di cui all'art. 17, comma 1, lettera j) dello Statuto Sociale}
	\begin{enumerate}
		\item Il Consiglio Direttivo ha facoltà di nominare, tra gli associati, anche tra soggetti esterni al Consiglio Direttivo, dei delegati a svolgere particolari funzioni stabilite di volta in  volta dal Consiglio stesso. Tali funzioni, a titolo esemplificativo e non esaustivo, sono, fra le tante, le seguenti:
		\begin{enumerate}
			\item Responsabile Informatico: coordina la manutenzione degli spazi web gestiti dalla Società e la pubblicazione sul sito internet istituzionale, e qualora adottati, sui social, sui canali video e sulle gallerie immagini di documenti e aggiornamenti.
			\item Direttore Tecnico: 
			\begin{enumerate}
				\item segue l'andamento di tutti i gruppi di allenamento.
				\item avendo una buona conoscenza di tutti gli Atleti della Società, ne valuta e segue il miglioramento individuale.
				\item definisce in accordo con la Società gli obiettivi stagionali e le attività formative per Atleti e Tecnici.
				\item segnala la necessità di nuove attrezzature.
			\end{enumerate}
			\item Responsabile per le politiche di safeguarding: ha il compito di implementare, monitorare e rafforzare le politiche di tutela all'interno della associazione sportiva, garantendo che vengano adottate pratiche sicure per proteggere Atleti, in particolare i minori, da ogni forma di abuso e discriminazione.
		\end{enumerate}
		\item In caso di organizzazione di gare e manifestazioni varie è gradita la collaborazione di tutti i soci in quanto tenuti a prendere parte all'attività e allo sviluppo della società. I soci possono presentare al progetti riguardanti gli scopi sociali e il Consiglio Direttivo ne valuterà la pertinenza, la validità e le eventuali possibilità di attuazione. Nessun socio può spendere il nome e far uso del logo dell'associazione senza averne titolo e senza previa autorizzazione del Consiglio Direttivo.
	\end{enumerate}
	
	\section{Comportamento nel Dojo}
	\begin{enumerate}
		\item Ferme restando le disposizioni degli altri documenti associativi (Statuto, Codice di condotta e MOG) si evidenzia che il Dojo è definito come lo spazio dove si svolgono gli allenamenti, comprendente l'area del tatami con le relative pertinenze consistenti in spogliatoi, servizi igienici, magazzini, corridoi e aree della struttura ospitante.
		\item Nel Dojo ci si deve attenere alle norme del codice di educazione proprie del Judo. Le principali sono le seguenti:
		\begin{enumerate}
			\item Per quanto riguarda i locali di pertinenza della palestra:
			\begin{itemize}
				\item L'accesso agli spogliatoi è consentito 15 minuti prima dell'inizio delle lezioni mentre dopo la fine delle lezioni la permanenza in detti locali può durare al massimo 15 minuti.
				\item Ogni judoka, è tenuto a portare rispetto verso coloro che usufruiranno dell'impianto dopo di lui, lasciando quindi gli spogliatoi in perfetto stato. I rifiuti vanno gettati negli appositi cestini.
				\item Si consiglia a tutti gli judoka di non lasciare oggetti di valore negli spogliatoi in quanto la A.S.D. DOJO Trieste non si assume alcuna responsabilità in caso di eventuali furti.
				\item Si accede ai locali del Dojo unicamente con scarpe pulite e vestiti adeguati.
				\item È necessario indossare sempre le ciabatte al di fuori del tatami, nell'uso delle docce (ove e se possibile) e negli spogliatoi.
				\item Gli indumenti lasciati negli spogliatoi devono essere riposti con ordine o sugli appositi appendiabiti o nella propria borsa in modo da evitare confusione o smarrimento degli stessi.
				\item Ogni judoka è responsabile di tutti gli ambienti del Dojo e deve vigilare anche sull'operato altrui, informando tempestivamente di ogni anomalia l'insegnante tecnico presente, al fine di evitare danneggiamenti alla struttura ospitante. 
				\item È obbligatorio segnalare tempestivamente agli istruttori dell'Associazione eventuali guasti nella struttura.
				\item Non è permesso introdurre o consumare cibo e alcolici all'interno dei locali del Dojo, fatta eccezione per particolari eventi organizzati dal Consiglio Direttivo o concordati con lo stesso. 
				\item Non è permesso introdurre nel Dojo, bottiglie in vetro, oggetti affilati ed appuntiti, materiale pirotecnico, infiammabile o qualsiasi altro materiale, apparecchiature od oggetti che potrebbero arrecare danni alla struttura.
			\end{itemize}
			\item Per quanto riguarda il tatami:
			\begin{itemize}
				\item Nella pratica del Judo viene vietato l'uso di indumenti diversi dal judogi od aggiuntivi ad esso, fatta eccezione per: gli indumenti intimi,  la maglietta obbligatoria per le atlete e  gli ausili medici. 
				\item Il judogi va indossato pulito e in ordine.
				\item Sul tatami si sale scalzi. È vietato calpestare il tatami con qualsiasi tipo di calzatura.
				\item È necessario osservare scrupolosamente l'igiene personale. 
				\item È inopportuno presentarsi agli allenamenti truccati e con le unghie lunghe. Le unghie delle mani e dei piedi devono essere corte per evitare di ferire gli altri durante l'allenamento e per poter svolgere correttamente la pratica del judo. 
				\item Non è permesso indossare bracciali, collane, anelli, orecchini, orologi e qualsiasi altro 	monile od oggetto durante lo svolgimento dell'allenamento. 
				\item È necessario usare le attrezzature e le infrastrutture in maniera adeguata e corretta. 
				\item È opportuno controllare il linguaggio ed evitare reazioni improvvise.
				\item Bisogna salutare con rispetto ogni persona che si incontra e rispondere gentilmente al saluto di chiunque.
				\item Bisogna mantenere una posizione composta  anche in momenti di riposo.
				\item Non bisogna disturbare ed intralciare con il proprio comportamento lo svolgimento degli allenamenti.
				\item Non è permesso adottare atteggiamenti inopportuni e non consoni allo svolgimento 	dell'attività sportiva.
			\end{itemize}
			\item In generale:
			\begin{itemize}
				\item In caso di malattia infettiva e/o contagiosa si invita a non frequentare gli allenamenti fino a guarigione avvenuta e possibilità di contagio esclusa. Considerata la particolare natura del Judo che prevede uno stretto contatto corpo a corpo, si invita a segnalare tempestivamente agli istruttori l'insorgenza di malattie "dermatologiche" (es. verruche, micosi, molluschi contagiosi, pediculosi e simili) al fine di consentire la sanificazione dei tatami e/o degli attrezzi e delle docce. Si chiede inoltre di restare lontani dal tatami fino ad avvenuta e certificata guarigione o, se i sanitari lo ritengono sufficiente a evitare il contagio, si chiede di coprire e isolare adeguatamente la parte del corpo interessata. Queste accortezze reciproche sono necessarie per tutelare la salute di tutti i judoka che salgono sul tatami.
			\end{itemize}
		\end{enumerate}
	\end{enumerate}
	
	\section{Tecnici}
	\begin{enumerate}
		\item I Tecnici sono delegati alla gestione delle attività sportive nel rispetto dello Statuto associativo, del presente Regolamento, del Codice di Condotta, del MOG, nonché delle norme Federali e di tutti i regolamenti sportivi che i Tecnici sono tenuti a far rispettare anche agli Atleti.
		\item In particolare i Tecnici devono:
		\begin{enumerate}
			\item prendersi cura degli Atleti che sono loro affidati conducendo l'allenamento di minorenni e maggiorenni dal momento in cui gli Atleti accedono al tatami per la lezione e fino alla sua conclusione; in particolare gli Atleti minorenni non possono mai essere lasciati da soli. Nel caso in cui il tecnico debba assentarsi momentaneamente dall'area di allenamento, può affidare i minori solo ad altri Tecnici e/o dirigenti sociali. Se queste figure non sono presenti può chiedere ad un esercente la responsabilità genitoriale o tutoria o a un accompagnatore autorizzato di vigilare momentaneamente i minori sospendendo ogni forma di attività sportiva che sarà ripresa solo con il ritorno e il benestare del tecnico. Gli allenamenti dei maggiorenni potranno anch'essi svolgersi solo in presenza del tecnico;
			\item sorvegliare l'incolumità degli Atleti, verificando che si allenino sempre in condizioni di sicurezza adeguate;
			\item verificare che gli Atleti siano in regola con la visita medico sportiva e sospendere l'attività se l'atleta non provvede a fornire la documentazione richiesta;
			\item allontanare dal tatami, mantenendoli al bordo dello stesso quegli Atleti che hanno un comportamento scorretto e ineducato nei confronti degli altri Atleti o del Tecnico stesso, segnalando se del caso la situazione, al Consiglio Direttivo o al Responsabile del Safeguarding in caso di sua specifica competenza.
		\end{enumerate}
	\end{enumerate}
	
	\section{Atleti}
	\begin{enumerate}
		\item Gli Atleti devono praticare l'attività sportiva nel rispetto dello Statuto associativo, del presente Regolamento, del Codice di Condotta, del MOG, nonché delle norme Federali e di tutti i regolamenti sportivi e tecnici.
		\item Gli Atleti devono:
		\begin{itemize}
			\item presentarsi regolarmente e puntualmente agli allenamenti;
			\item segnalare tempestivamente ai Tecnici di riferimento la propria assenza dalle sedute di allenamento o l'impossibilità di partecipare a una manifestazione (sia questa di carattere agonistico, promozionale, benefico, stage di allenamento, riunioni formative e/o informative) in cui si è stati convocati. La segnalazione per i minorenni verrà effettuata dall'esercente la responsabilità genitoriale;
			\item segnalare ai Tecnici di riferimento i motivi della propria assenza prolungata dagli allenamenti. La segnalazione per i minorenni verrà effettuata dall'esercente la responsabilità genitoriale;
			\item segnalare al Tecnico di riferimento ogni problema di natura fisica, prima o durante la seduta di allenamento. La segnalazione per i minorenni prima dell'allenamento verrà effettuata dall'esercente la responsabilità genitoriale;
			\item rispettare i tempi della convalescenza prescritti dai sanitari in caso di malattia o di infortunio e riprendere gli allenamenti dopo aver presentato certificazione medica attestante la guarigione;
			\item segnalare ai medici curanti l'esercizio di attività sportiva agonistica, per evitare la prescrizione di farmaci dopanti (lista WADA). La segnalazione per i minorenni verrà effettuata dall'esercente la responsabilità genitoriale;
			\item segnalare al Tecnico di riferimento l'utilizzo regolare o l'assunzione improvvisa di farmaci, in prossimità di un impegno agonistico, per verificare l'esclusione dei farmaci prescritti dai medici dalla lista dei farmaci dopanti (lista WADA);
			\item evitare l'uso di qualsiasi sostanza che alteri il proprio equilibrio psicofisico;
			\item se un atleta vuole seguire uno o più allenamenti presso una palestra diversa dalla propria deve concordare questa attività con il tecnico che gli darà o meno il permesso e in caso di consenso si attiverà per i documenti e i permessi necessari. 
			\item frequentare con regolarità gli allenamenti, condizione questa indispensabile per il passaggio di cintura;
			\item essere consapevoli che dalla cintura verde in poi gli avanzamenti di grado saranno fondati,  oltre che sull'assidua frequenza agli allenamenti e sul comportamento nel Dojo, sulle effettive conoscenze raggiunte. Tale scelta è motivata dal valore che una cintura blu o marrone riveste nella maturazione sportiva e personale di un allievo.
			\item non rifiutare mai di praticare il Judo con un compagno di allenamento, perché nel Judo la disponibilità verso l’altro è completa. 
			\item rispettare il livello tecnico di ciascuno essendo responsabili dell’incolumità e del benessere del compagno d’allenamento
			\item non dimostrare o eseguire le tecniche di Judo al di fuori del Dojo.
		\end{itemize}
	\end{enumerate}
	
	\section{Esercenti la responsabilità genitoriale e accompagnatori}
	\begin{enumerate}
		\item Gli accompagnatori sono soggetti autorizzati, tramite modulo predisposto dall'ASD DOJO Trieste, dall'esercente la responsabilità genitoriale (o tutoria) ad accompagnare e/o prelevare il minore.
		\item Gli esercenti la responsabilità genitoriale e gli accompagnatori devono mantenere in ogni circostanza (allenamenti ai quali periodicamente possono assistere, gare, riunioni ecc.) un comportamento corretto e consono ai principi etici e sportivi del Judo (rispetto, coraggio, sincerità, onore, modestia, gratitudine, autocontrollo e amicizia)
		\begin{itemize}
			\item Gli esercenti la responsabilità genitoriale e gli accompagnatori degli Atleti minorenni, sono tenuti a vigilare sul comportamento dei minori in loro custodia sino all'ingresso degli spogliatoi prima dell'allenamento e all'uscita degli spogliatoi dopo l'allenamento.
			\item L'eventuale ingresso di esercenti la potestà genitoriale o tutoria ovvero preposti alla vigilanza (accompagnatori) è regolamentato dal MOG del 22.08.2024 e da sue eventuali integrazioni e modificazioni.
			\item I bambini/ragazzi fino ai 12 anni, abbandoneranno la palestra solo in compagnia degli esercenti la responsabilità genitoriale o tutoria o accompagnatori autorizzati a cui saranno consegnati dagli insegnanti.
			\item I ragazzi dai 13 ai 18 anni per i quali non è stato sottoscritto dall'esercente la responsabilità genitoriale  o tutoria il permesso di uscita in autonomia dal Dojo, abbandoneranno la palestra solo in compagnia dell' esercente la responsabilità genitoriale o tutoria o dell'accompagnatore autorizzato.
		\end{itemize}
		\item Gli allenamenti si svolgono a porte chiuse e gli esercenti la responsabilità genitoriale o tutoria, gli accompagnatori autorizzati possono assistervi una volta al mese, in particolare durante la prima lezione del mese del corso frequentato dal minore.
	\end{enumerate}
	
	\section{Provvedimenti disciplinari}
	\begin{enumerate}
		\item Il Consiglio Direttivo, può irrogare sanzioni disciplinari nei confronti di un socio nel caso in cui:
		\begin{itemize}
			\item il socio violi lo Statuto,  il Regolamento sociale, il Codice di Condotta e il Modello Organizzativo e di Controllo dell'attività sportiva (c.d. MOG);
			\item il socio sia stato destinatario di provvedimenti disciplinari inflitti dagli organi competenti del CONI, della FIJLKAM, delle Federazioni Sportive Nazionali, delle Discipline Sportive Associate, degli Enti di Promozione Sportiva o di organismi sportivi internazionali riconosciuti e, comunque, ogniqualvolta il comportamento dell'associato comprometta o rischi di pregiudicare il prestigio o l'immagine dell'Associazione.
		\end{itemize}
		\item I provvedimenti disciplinari irrogabili con le formalità previste ai commi successivi sono i seguenti:
		\begin{itemize}
			\item ammonizione scritta;
			\item sospensione fino ad un massimo di sei mesi da qualsiasi attività sociale;
			\item espulsione automatica dall'Associazione per morosità protrattasi per un periodo di oltre tre mesi decorrenti dall'inizio dell'esercizio sociale (art. 6 comma 3 Statuto);
			\item espulsione dall'associazione nel caso di gravi violazioni delle regole associative e dei principi e valori fondativi dell'Associazione (art. 6, comma 4 Statuto). 
		\end{itemize}
		\item Tutte le decisioni disciplinari sono assunte col voto favorevole della maggioranza dei componenti del Consiglio Direttivo con deliberazione motivata entro 30 (trenta) giorni dalla realizzazione o dell'avvenuta conoscenza della condotta o dalla comunicazione di provvedimenti disciplinari di cui al punto 1. b). Tutte  decisioni disciplinari  del Consiglio Direttivo sono comunicate all'interessato a mezzo di posta elettronica certificata o a mezzo raccomandata a.r o a mezzo mail semplice.
		\item Contro il provvedimento di irrogazione delle sanzioni indicate ai punti a) e b) del comma 2 del presente articolo il socio può ricorrere allo stesso Consiglio Direttivo presentando, a pena di decadenza, entro e non oltre 15 (quindici) giorni  dalla comunicazione della sanzione, la propria impugnazione da inviarsi a mezzo PEC o mail. La presentazione dell'impugnazione sospende l'esecuzione del provvedimento disciplinare e in questa fase della procedura non potranno essere addotti nuovi fatti o nuovi addebiti a carico del socio per la medesima condotta. La decisione definitiva deve essere disposta entro il termine massimo di 30 (trenta) giorni dalla ricezione dell'impugnazione e nel medesimo termine, solo dietro eventuale richiesta formulata dall'incolpato in seno alle osservazioni trasmesse, può esserne disposta l'audizione. Nel corso dell'audizione il socio può farsi assistere da altro socio.
		\item Contro il provvedimento di irrogazione delle sanzione indicata al p.to c) del comma 2 del presente articolo  non è ammessa impugnazione, trattandosi di espulsione automatica per mancato pagamento della quota associativa.
		\item Contro il provvedimento di irrogazione delle sanzione indicata al p.to d) del comma 2 del presente articolo l'interessato può proporre reclamo all'Assemblea generale, a pena di decadenza, entro e non oltre 15 (quindici) giorni dal ricevimento del provvedimento. La discussione sull'esclusione avviene alla prima Assemblea utile successiva al ricevimento del ricorso. Il provvedimento di esclusione rimane sospeso fino alla decisione dell'Assemblea. L'esclusione diventa operativa con l'annotazione del provvedimento nel libro dei soci che avviene decorsi  30 (trenta) giorni dal ricevimento del provvedimento ovvero a seguito di delibera dell'Assemblea che abbia ratificato il provvedimento di esclusione adottato dal Consiglio Direttivo.
		\item Qualora un'infrazione disciplinare dovesse essere contestata ad un membro del Consiglio Direttivo, il procedimento si svolge in unico grado avanti Consiglio Direttivo: in tal caso il voto palese del Presidente prevale in caso di parità dei voti.
		\item Qualora un'infrazione disciplinare dovesse essere contestata al Presidente dell'Associazione, la medesima procedura descritta al comma precedente sarà attuata dal Vicepresidente.
	\end{enumerate}
	
	\section*{Norme finali}
	Il presente regolamento è affisso presso la bacheca sociale e viene pubblicato sul sito internet istituzionale. I soci sono invitati a leggerlo attentamente.
	
\end{document}