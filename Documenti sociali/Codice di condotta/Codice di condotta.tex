% !TeX spellcheck = it_IT
% !TeX encoding = UTF-8
\documentclass{djtsdoc}

\title{CODICE DI CONDOTTA \\[10pt] {\normalfont\large per la tutela dei minori e per la prevenzione delle molestie, della violenza di genere e di ogni altra condizione di discriminazione, c.d. politiche di Safeguarding
dell'Associazione Sportiva Dilettantistica DOJO Trieste}}
\date{Trieste, 22 agosto 2024}

\usepackage{titlesec}
\titleformat{\section}{\normalfont\large\bfseries}{Art. \thesection\ -\ }{0pt}{}
\titlespacing*{\section}{0pt}{0.4in}{0.2in}

\begin{document}
		
	\maketitle
	
	\section*{Premessa}
	L'Associazione Sportiva Dilettantistica DOJO Trieste, i suoi Tesserati e le sue Tesserate, nelle rispettive qualità di Affiliata e Tesserati FIJLKAM, si conformano, unitamente alla Federazione stessa, alle disposizioni di cui ai D. Lgs. n. 36 e 39 del 28 febbraio 2021, nonché alle disposizioni emanate dalla Giunta Nazionale del CONI e dall'Osservatorio permanente del CONI per le Politiche di \textit{Safeguarding} in materia, nonché al Regolamento per la tutela dei Tesserati - di \textit{Safeguarding Policy} approvato dal Consiglio Federale FIJLKAM in data 1 dicembre 2023 e alle Linee Guida Federali pubblicate il 31 agosto 2023, in base alle quali viene emanato il seguente Codice per l'adozione di ogni necessaria misura volta a favorire il pieno sviluppo fisico, emotivo, intellettuale e sociale dell'Atleta, la sua effettiva partecipazione all'attività sportiva nonché la piena consapevolezza di tutti i Tesserati e le Tesserate in ordine ai propri e altrui diritti, doveri, obblighi, responsabilità e tutele.
	
	\section{Principi}
	\begin{enumerate}
		\item L'ASD DOJO Trieste riconosce e garantisce il diritto di tutti i Tesserati e le Tesserate a essere trattati con rispetto e dignità.
		\item L'ASD DOJO Trieste riconosce e garantisce la tutela di tutti i Tesserati e le Tesserate contro ogni forma di abuso, molestia, violenza di genere e ogni altra condizione di discriminazione, prevista dal Decreto Legislativo 11 aprile 2006, n. 198.
		\item L'ASD DOJO Trieste riconosce e garantisce la piena tutela del diritto alla salute e al benessere psico-fisico dei Tesserati e delle Tesserate, con particolare riguardo per i minori, quale valore preminente e assolutamente prevalente anche rispetto al risultato sportivo.
		\item Nel riconoscimento dei diritti e delle tutele invocate, l'ASD DOJO Trieste riconosce parità di trattamento dei Tesserati e delle Tesserate indipendentemente da etnia, convinzioni personali, disabilità, età, identità di genere, orientamento sessuale, lingua, opinione politica, religione, condizione patrimoniale, di nascita, fisica, intellettiva, relazionale o sportiva.
	\end{enumerate}
	
	\section{Ambito di applicazione}
	\begin{enumerate}
		\item Il presente codice si applica a tutti i Tesserati e le Tesserate dell'ASD DOJO Trieste nonché ai lavoratori, collaboratori e volontari e in generale gli operatori sportivi che, nel contesto del sodalizio a qualsiasi titolo e in qualsiasi ruolo, sono a contatto con gli Atleti o che in ogni caso sono coinvolti nell'attività sportiva.
	\end{enumerate}
	
	\section{Obiettivi e finalità}
	\begin{enumerate}
		\item Obiettivo della ASD DOJO Trieste, nel rispetto dei generali principi di lealtà, probità e correttezza, è quello di tutelare i minori, prevenire le molestie, la violenza di genere e ogni altra forma di discriminazione, attraverso strumenti, attuati in ossequio alle disposizioni del presente codice e anche in base al modello organizzativo e di controllo, finalizzati:
		\begin{enumerate}
			\item all'educazione alla formazione e allo svolgimento di una pratica sportiva sana;
			\item alla piena consapevolezza di tutti i Tesserati e le Tesserate in ordine ai propri diritti, doveri,	obblighi, responsabilità e tutele;
			\item alla creazione di un ambiente sano, sicuro e inclusivo che garantisca la dignità,	l'uguaglianza, l'equità e il rispetto dei diritti dei Tesserati e delle Tesserate in particolare se	minori;
			\item alla valorizzazione delle diversità;
			\item alla promozione del pieno sviluppo della persona-atleta, in particolare se minore;
			\item alla promozione, da parte di Dirigenti e Tecnici, del benessere dell'Atleta;
			\item alla effettiva partecipazione di tutti i Tesserati e le Tesserate all'attività sportiva secondo le rispettive aspirazioni, potenzialità, capacità e specificità;
			\item alla prevenzione e al contrasto di ogni forma di abuso, violenza e discriminazione;
			\item alla rimozione degli ostacoli che impediscano:
			\begin{enumerate}
				\item la promozione del benessere dell'Atleta, in particolare se minore, e dello sviluppo 	psico-fisico dello stesso secondo le relative aspirazioni, potenzialità, capacità e specificità;
				\item la partecipazione dell'Atleta alle attività, indipendentemente da etnia, convinzioni personali, disabilità, età, identità di genere, orientamento sessuale, lingua, opinione politica, religione, condizione patrimoniale, di nascita, fisica, intellettiva, relazionale o sportiva.
			\end{enumerate}
		\end{enumerate}
	\end{enumerate}
	
	\section{Diritti, doveri e obblighi a carico di tutti i Tesserati e le Tesserate}
	\begin{enumerate}
		\item A tutti i Tesserati e alle Tesserate sono riconosciuti i diritti fondamentali:
		\begin{enumerate}
			\item a un trattamento dignitoso e rispettoso in ogni rapporto, contesto, situazione, attività ed evento nell'ambito del sodalizio sportivo e in genere dell'attività federale;
			\item alla tutela da ogni forma di abuso, molestia, violenza di genere e ogni altra condizione di	discriminazione, indipendentemente da etnia, convinzioni personali, disabilità, età, identità di	genere, orientamento sessuale, lingua, opinione politica, religione, condizione patrimoniale, di 	nascita, fisica, intellettiva, relazionale o sportiva;
			\item alla garanzia che la salute e il benessere psico-fisico siano prevalenti rispetto a ogni risultato sportivo.
		\end{enumerate}
		\item Coloro che prendono parte, a qualsiasi titolo e in qualsiasi funzione e/o ruolo, all'attività sportiva, in forma diretta o indiretta, sono tenuti a rispettare tutte le disposizioni e le prescrizioni a tutela degli indicati diritti dei Tesserati e delle Tesserate.
		\item Tutti i Tesserati e le Tesserate sono tenuti a:
		\begin{enumerate}
			\item comportarsi secondo lealtà, probità e correttezza nello svolgimento di ogni attività connessa o	collegata all'ambito sportivo e tenere una condotta improntata al rispetto nei confronti degli	altri Tesserati e delle altre Tesserate;
			\item astenersi dall'utilizzo di un linguaggio, anche corporeo, inappropriato o allusivo, anche in	situazioni ludiche, per gioco o per scherzo;
			\item garantire la sicurezza e la salute degli altri Tesserati e delle altre Tesserate, impegnandosi a	creare e a mantenere un ambiente sano, sicuro e inclusivo;
			\item impegnarsi nell'educazione e nella formazione della pratica sportiva sana, supportando gli 	altri Tesserati e le altre Tesserate nei percorsi educativi e formativi;
			\item impegnarsi a creare, mantenere e promuovere un equilibrio sano tra ambito personale e 	sportivo, valorizzando anche i profili ludici, relazionali e sociali dell'attività sportiva;
			\item instaurare un rapporto equilibrato con coloro che esercitano la responsabilità genitoriale o i	soggetti cui è affidata la cura degli Atleti e delle Atlete ovvero loro delegati;
			\item prevenire e disincentivare dispute, contrasti e dissidi anche mediante l'utilizzo di una	comunicazione sana, efficace e costruttiva;
			\item affrontare in modo proattivo comportamenti offensivi, manipolativi, minacciosi o aggressivi;
			\item collaborare con gli altri Tesserati e le altre Tesserate nella prevenzione, nel contrasto e nella	repressione di abusi, violenze e discriminazioni (individuali o collettivi);
			\item segnalare senza indugio al Responsabile delle politiche di \textit{Safeguarding} della ASD DOJO Trieste (art. 8) situazioni, anche potenziali, che espongano sé o altri a pregiudizio, pericolo, timore o disagio.
		\end{enumerate}
	\end{enumerate}
	
	\section{Doveri e obblighi a carico dei Dirigenti Sportivi e degli Insegnanti Tecnici}
	\begin{enumerate}
		\item Tutti i Dirigenti sportivi e gli Insegnanti Tecnici sono tenuti a:
		\begin{enumerate}
			\item agire per prevenire e contrastare ogni forma di abuso, violenza e discriminazione;
			\item astenersi da qualsiasi abuso o uso improprio della propria posizione di fiducia, potere o 	influenza nei confronti dei Tesserati e delle Tesserate, specie se minori;
			\item contribuire alla formazione e alla crescita armonica dei Tesserati e delle Tesserate, in 	particolare	se minori;
			\item evitare ogni contatto fisico non necessario con i Tesserati e le Tesserate, in particolare se 	minori;
			\item promuovere un rapporto tra tutti i Tesserati e le Tesserate improntato al rispetto e alla	collaborazione, prevenendo situazioni disfunzionali, che creino, anche mediante manipolazione, uno stato di soggezione, pericolo o timore;
			\item astenersi dal creare situazioni di intimità con il Tesserato e la Tesserata minore;
			\item porre in essere, in occasione delle trasferte, soluzioni logistiche atte a prevenire situazioni di disagio e/o comportamenti inappropriati, coinvolgendo nelle scelte coloro che esercitano la	responsabilità genitoriale o i soggetti cui è affidata la loro cura ovvero loro delegati;
			\item comunicare e condividere con il Tesserato e la Tesserata minore gli obiettivi educativi e	formativi,	illustrando le modalità con cui si intendono perseguire tali obiettivi e	coinvolgendo nelle scelte coloro che esercitano la responsabilità genitoriale o i soggetti cui è affidata la loro cura ovvero loro delegati;
			\item astenersi da comunicazioni e contatti di natura intima con il Tesserato e la Tesserata minore, 	anche mediante social network e canali di comunicazione a distanza o di messaggistica rapida;
			\item interrompere senza indugio ogni contatto con il Tesserato e la Tesserata minore qualora si 	riscontrino situazioni di ansia, timore o disagio derivanti dalla propria condotta, attivando il 	Responsabile delle politiche di \textit{Safeguarding} della ASD DOJO Trieste (art.8);
			\item impiegare le necessarie competenze professionali nell'eventuale programmazione e/o gestione 	di regimi alimentari in ambito sportivo, ferma restando la possibilità per ogni Atleta di	provvedervi autonomamente;
			\item segnalare tempestivamente eventuali indicatori di disturbi alimentari degli Atleti e delle Atlete loro affidati;
			\item dichiarare all'organo direttivo della ASD DOJO Trieste la sussistenza o la sopravvenienza di	cause di incompatibilità e/o di conflitti di interesse;
			\item sostenere i valori del sport, altresì educando al ripudio di sostanze o metodi vietati per alterare le prestazioni sportive dei Tesserati e delle Tesserate;
			\item conoscere, informarsi e aggiornarsi con continuità sulle politiche di \textit{Safeguarding}, sulle misure di prevenzione e contrasto agli abusi, violenze e discriminazioni, nonché sulle più moderne metodologie di formazione e comunicazione in ambito sportivo;
			\item astenersi dall'utilizzo, dalla riproduzione e dalla diffusione di immagini o video dei Tesserati e delle Tesserate minori, se non per finalità educative e formative, acquisendo in ogni caso le 	necessarie autorizzazioni da coloro che esercitano la responsabilità genitoriale o dai soggetti cui è affidata la loro cura ovvero da loro delegati;
			\item segnalare senza indugio al Responsabile dell'Affiliata delle politiche di \textit{Safeguarding} (art.8) situazioni, anche potenziali, che espongano i Tesserati e le Tesserate a pregiudizio, pericolo, timore o disagio.
		\end{enumerate}
	\end{enumerate}
	
	\section{Diritti, doveri e obblighi degli Atleti e delle Atlete}
	\begin{enumerate}
		\item Tutti gli Atleti e le Atlete sono tenuti a:
		\begin{enumerate}
			\item rispettare il principio di solidarietà tra Atleti e Atlete, favorendo assistenza e sostegno reciproco;
			\item comunicare le proprie aspirazioni ai Dirigenti Sportivi e ai Tecnici e valutare in spirito di	collaborazione le proposte circa gli obiettivi educativi e formativi e le modalità di 	raggiungimento di tali obiettivi, anche con il supporto di coloro che esercitano la responsabilità genitoriale o dei soggetti cui è affidata la loro cura, eventualmente confrontandosi con gli altri Atleti e le altre Atlete;
			\item comunicare ai Dirigenti Sportivi ed ai Tecnici situazioni di ansia, timore o disagio che 	riguardino sé o altri;
			\item prevenire, evitare e segnalare situazioni disfunzionali che creino, anche mediantemanipolazione, uno stato di soggezione, pericolo o timore negli altri Atleti e nelle altre Atlete;
			\item rispettare e tutelare la dignità, la salute e il benessere degli altri Atleti e delle altre Atlete e, più 	in generale, di tutti i soggetti coinvolti nelle attività sportive;
			\item rispettare la funzione educativa e formativa dei Dirigenti Sportivi e dei Tecnici;
			\item mantenere rapporti improntati al rispetto con gli altri Atleti e con le altre Atlete e con ogni	soggetto comunque coinvolto nelle attività sportive;
			\item riferire qualsiasi infortunio o incidente agli esercenti la responsabilità genitoriale o ai soggetti	cui è affidata la cura degli Atleti e delle Atlete ovvero ai loro delegati;
			\item evitare contatti e situazioni di intimità con Dirigenti Sportivi e Tecnici, anche in occasione di	trasferte, segnalando eventuali comportamenti inopportuni;
			\item astenersi dal diffondere materiale fotografico e video di natura privata o intima proprio o altrui, anche ricevuto da terzi, segnalando comportamenti difformi a coloro che esercitano la	responsabilità genitoriale o ai soggetti cui è affidata la loro cura ovvero ai loro delegati, nonché al Responsabile delle politiche di \textit{Safeguarding} della ASD DOJO Trieste (art.8);
			\item segnalare senza indugio al Responsabile delle politiche di \textit{Safeguarding} della ASD DOJO	Trieste (art.8) situazioni, anche potenziali, che espongano sé o altri a pericolo o pregiudizio.
		\end{enumerate}
	\end{enumerate}
	
	\section{Fattispecie}
	\begin{enumerate}
		\item Per la salvaguardia e la tutela dei Tesserati e delle Tesserate, costituiscono condotte rilevanti ai fini della presente normativa relativa alle politiche di \textit{Safeguarding} le seguenti fattispecie:
		\begin{enumerate}
			\item \textbf{l'abuso psicologico}: qualunque atto indesiderato, tra cui la mancanza di rispetto, il	confinamento, la sopraffazione, l'isolamento o qualsiasi altro trattamento che possa incidere	sul senso di identità, dignità e autostima, ovvero tale da intimidire, turbare o alterare la serenità del Tesserato/della Tesserata, anche se perpetrato attraverso l'utilizzo di strumenti digitali;
			\item \textbf{l'abuso fisico}: qualunque condotta consumata o tentata (tra cui botte, pugni, percosse, soffocamento, schiaffi, calci o lancio di oggetti), che sia potenzialmente in grado di procurare 	direttamente o indirettamente un danno alla salute, un trauma, delle lesioni fisiche o che	danneggi lo sviluppo psico-fisico del minore tanto da compromettergli una sana e serena crescita. Tali atti possono anche consistere nell'indurre un Tesserato/una Tesserata a svolgere (al fine di una migliore performance sportiva) un'attività fisica inappropriata, come il somministrare carichi di allenamento inadeguati in base all'età, genere, struttura e capacità fisica oppure forzare ad allenarsi Atleti ammalati, infortunati o comunque doloranti, nonché nell'uso improprio, eccessivo, illecito o arbitrario di strumenti sportivi. In quest'ambito rientrano anche quei comportamenti che favoriscono il consumo di alcool, di sostanze comunque vietate da norme vigenti o le pratiche di doping;
			\item \textbf{la molestia sessuale}: qualunque atto o comportamento indesiderato e non gradito di natura sessuale, sia esso verbale, non verbale o fisico che comporti uno stato di sofferenza fisica e/o psicologica, anche solo generando grave disappunto, fastidio, disturbo, disgusto. Tali atti o	comportamenti possono anche consistere nell'assumere un linguaggio del corpo inappropriato, nel rivolgere osservazioni o allusioni sessualmente esplicite, nonché richieste indesiderate o non gradite aventi connotazione sessuale, ovvero telefonate, messaggi, lettere od ogni altra forma di comunicazione a contenuto sessuale, anche con effetto intimidatorio, degradante o umiliante;
			\item \textbf{l'abuso sessuale}: qualsiasi comportamento o condotta avente connotazione sessuale, con o	senza contatto, considerata non desiderata, o il cui consenso è estorto, costretto, manipolato,	non dato o negato. Può consistere anche nel costringere un Tesserato/una Tesserata a porre in	essere condotte sessuali inappropriate o indesiderate o nell'osservare, anche di nascosto, il	Tesserato /la Tesserata in condizioni e contesti intimi e/o non appropriati;
			\item \textbf{la negligenza}: il mancato intervento di un esponente federale (Dirigente, Tecnico o qualsiasi soggetto tesserato), anche in ragione dei doveri che derivano dal suo ruolo, che, presa	conoscenza di uno degli eventi o comportamento o condotta o atto di cui al presente documento, omette di intervenire con ciò causando un danno, permettendo che venga causato	un danno o creando un pericolo imminente di danno. Può consistere anche nel persistente e sistematico disinteresse, ovvero trascuratezza, dei bisogni fisici e/o psicologici del Tesserato/della Tesserata;
			\item \textbf{l'incuria}: la mancata soddisfazione delle necessità fondamentali a livello fisico, medico,	educativo ed emotivo;
			\item \textbf{l'abuso di matrice religiosa}: l'impedimento, il condizionamento o la limitazione del diritto di professare liberamente la propria fede religiosa e di esercitarne in privato o in pubblico il culto purché non si tratti di riti contrari al buon costume;
			\item \textbf{il bullismo, il cyberbullismo}: qualsiasi comportamento offensivo e/o aggressivo che un	singolo individuo o più soggetti possono mettere in atto, personalmente, attraverso i social	network o altri strumenti di comunicazione, sia in maniera isolata, sia ripetutamente nel corso del tempo, ai danni di uno o più Tesserati/Tesserate, con lo scopo di esercitare nei suoi /loro confronti un potere o un dominio. Possono anche consistere in comportamenti di	prevaricazione e sopraffazione ripetuti e atti a intimidire o turbare un soggetto Tesserato che determinano una condizione di disagio, insicurezza, paura, esclusione o isolamento (tra cui umiliazioni, critiche riguardanti l'aspetto fisico, minacce verbali, anche in relazione alla performance sportiva, diffusione di notizie infondate, minacce di ripercussioni fisiche o di danneggiamento di oggetti posseduti dalla vittima);
			\item \textbf{i comportamenti discriminatori}: qualsiasi comportamento finalizzato a conseguire un effetto discriminatorio basato su etnia, colore, caratteristiche fisiche, genere, status social-economico, prestazioni sportive, capacità atletiche, religione, convinzioni personali, disabilità, età, orientamento sessuale;
			\item \textbf{l'abuso dei mezzi di correzione e/o disciplina anche nell'attività di preparazione e allenamento}: la condotta che, trascendendo i limiti dell'uso del potere correttivo e disciplinare spettante a un Tecnico o un Dirigente nei confronti della persona offesa, venga esercitato con modalità non adeguate rispetto alle condizioni proprie dell'Atleta e/o al fine/risultato sportivo da raggiungere, o allo scopo di perseguire un interesse diverso da quello per il quale tale potere è conferito dall'ordinamento federale.
		\end{enumerate}
	\end{enumerate}
	
	\section{Responsabile del sodalizio affiliato contro abusi, violenze e discriminazioni}
	\begin{enumerate}
		\item Allo scopo di prevenire e contrastare ogni tipo di abuso, violenza e discriminazione sui Tesserati e sulle Tesserate nonché per garantire la protezione dell'integrità fisica e morale degli sportivi, l'organo direttivo dell'ASD DOJO Trieste ha nominato con delibera del Consiglio Direttivo del 28 giugno 2024 un responsabile contro abusi, violenze e discriminazioni, il c.d. Responsabile per le politiche di \textit{Safeguarding} dell'ASD DOJO Trieste, anche ai sensi dell'art. 33, comma 6, del d.lgs. n. 36 del 28 febbraio 2021, giusta delibera della Giunta Nazionale del CONI del 25 luglio 2023, n. 255.
		\item La nomina del Responsabile per le politiche di \textit{Safeguarding} della ASD DOJO Trieste di cui al comma 1 è stata senza indugio: pubblicata sulla \textit{homepage} del sito della ASD DOJO Trieste e/o sui social network facenti capo al sodalizio; conservata agli atti presso la sua sede, affissa nell'impianto sportivo in uso e comunicata al \textit{Safeguarding Office} della Federazione.
	\end{enumerate}
	
	\section{Selezione degli operatori sportivi}
	\begin{enumerate}
		\item Nella selezione dei candidati per le funzioni di operatori sportivi (tra cui Insegnanti Tecnici, Accompagnatori, Preparatori atletici, Massaggiatori, Medici sociali) al fine di garantire che siano idonei a operare nell'ambito delle attività giovanili e in diretto contatto con i Tesserati e le Tesserate minori, l'organo direttivo della ASD DOJO Trieste procederà:
		\begin{enumerate}
			\item a un colloquio preliminare con il candidato in merito alle tematiche di \textit{Safeguarding}, alla	presenza anche del Responsabile per le politiche di \textit{Safeguarding} del sodalizio;
			\item alla verifica presso gli uffici federali della sussistenza di precedenti disciplinari, a carico del candidato, nelle ipotesi previste dal presente codice e dalla normativa in materia di politiche \textit{Safeguarding};
			\item all'acquisizione obbligatoria delle idonee certificazioni rilasciate da parte delle autorità competenti relative ai precedenti penali del candidato.
		\end{enumerate}
	\end{enumerate}
	
	\section{Verifiche periodiche}
	\begin{enumerate}
		\item Almeno una volta per ogni anno sociale successivo a quello in cui è sorto il rapporto con l'operatore sportivo, l'ASD DOJO Trieste è tenuta ad acquisire, in forma di autodichiarazione, l'aggiornamento sullo stato dei carichi pendenti penali e disciplinari.
		\item Le dichiarazioni false rese alla ASD DOJO Trieste verranno valutate, a ogni effetto, alla stregua della fattispecie di cui il soggetto sia reso responsabile.
		\item In seguito all'instaurazione del rapporto con l'operatore sportivo  l'ASD DOJO Trieste è tenuta ad acquisire con cadenza semestrale le idonee certificazioni (di cui all'art. 9. 4) rilasciate da parte delle autorità competenti relative ai precedenti penali del operatore .
	\end{enumerate}
	
	\section{Conservazione documenti}
	\begin{enumerate}
		\item La documentazione e le informazioni acquisite nell'ambito delle attività previste egli articoli precedenti, sono accessibili esclusivamente al rappresentante legale del sodalizio, al personale dello stesso all'uopo delegato e al Responsabile per le politiche di \textit{Safeguarding}.
		\item Il supporto (cartaceo, digitale) contenente il materiale di cui al primo comma, rimane opportunamente custodito presso la sede della ASD DOJO Trieste  nel rispetto della normativa vigente.
	\end{enumerate}
	
	\section{Informazione}
	\begin{enumerate}
		\item L'ASD DOJO Trieste si impegna a diffondere l'adozione del presente codice nonché dei protocolli adottati attraverso i modelli organizzativi di controllo dell'attività sportiva mediante:
		\begin{itemize}
			\item pubblicazione sul proprio sito istituzionale, mediante accesso dalla homepage, del presente	codice, dei modelli organizzativi di controllo dell'attività sportiva e delle eventuali modifiche;
			\item pubblicazione e diffusione nei propri profili sui social network, del presente codice, dei	modelli organizzativi di controllo dell'attività sportiva e delle eventuali modifiche;
			\item invio a mezzo mail e/o PEC al momento dell'atto di sottoscrizione del tesseramento, a	qualsiasi titolo e in qualsiasi qualità, del testo del presente codice e dello schema dei modelli	organizzativi di controllo dell'attività sportiva nonché all'atto di stipula di qualsiasi rapporto	con gli operatori sportivi: la risposta alla mail con la dicitura \textit{"Confermo la ricevuta della mail e dei relativi allegati"} e/o con dicitura avente identico significato varrà come accettazione e	come quietanza della ricezione della documentazione ricevuta. La ricevuta di avvenuta	consegna della PEC varrà come accettazione e come quietanza;
			\item invio a mezzo mail e/o PEC a tutti i Tesserati, a tutte le Tesserate e a tutti gli operatori sportivi dei suddetti documenti in caso di modifiche apportate agli stessi in costanza di	rapporto. La risposta alla mail con la dicitura \textit{"Confermo la ricevuta della mail e dei relativi allegati"} e/o con dicitura avente identico significato varrà come accettazione e come quietanza della ricezione della documentazione ricevuta. La ricevuta di avvenuta consegna della PEC varrà come accettazione e come quietanza;
			\item consegna cartacea al momento dell'atto di sottoscrizione del tesseramento, a qualsiasi titolo e in qualsiasi qualità, del testo del presente codice e dello schema dei modelli organizzativi di 	controllo dell'attività sportiva nonché all'atto di stipula di qualsiasi rapporto con gli operatori 	sportivi: la sottoscrizione varrà come accettazione e come quietanza della ricezione della 	documentazione ricevuta;
			\item consegna cartacea a tutti i Tesserati, a tutte le Tesserate e a tutti gli operatori sportivi dei	suddetti documenti in caso di modifiche apportate agli stessi in costanza di rapporto, con	contestuale sottoscrizione che varrà come accettazione e come quietanza della ricezione della 	documentazione ricevuta;
			\item affissione presso l'impianto sportivo in uso.
		\end{itemize}
	\end{enumerate}
	
	\section{Formazione e aggiornamento}
	\begin{enumerate}
		\item Annualmente, tutti i soggetti coinvolti nelle attività sportive e relative ai Tesserati e alle Tesserate minori, della ASD DOJO Trieste dovranno frequentare corsi formazione e aggiornamento organizzati all'uopo e di cui la ASD DOJO Trieste dovrà dare adeguata informazione.
		\item I corsi potranno essere organizzati dalla ASD DOJO Trieste e anche dalla Federazione a livello centrale e a livello periferico anche attraverso le Strutture Territoriali.
	\end{enumerate}
	
	\section{Incompatibilità e conflitti di interesse}
	\begin{enumerate}
		\item Il rappresentante legale e gli operatori sportivi della ASD DOJO Trieste direttamente coinvolti nell'attività con i Tesserati e le Tesserate minori, sono incompatibili con la funzione di Responsabile per le politiche di \textit{Safeguarding} in ogni struttura sportiva.
		\item Eventuali confitti di interesse in materia, che non trovino un naturale e tempestivo componimento nel contesto della ASD DOJO Trieste saranno devoluti, per ogni opportuno provvedimento, al Responsabile per le politiche di \textit{Safeguarding} istituito presso la Federazione.
	\end{enumerate}
	
	\section{Procedure e sanzioni}
	\begin{enumerate}
		\item I soggetti che pongano in essere i comportamenti riconducibili alle fattispecie dei cui all'art. 7 del presente codice saranno sottoposti al procedimento sanzionatorio nell'ambito del medesimo sodalizio, ai sensi delle norme dello Statuto e del Regolamento della ASD DOJO Trieste;
		\item Ove la prosecuzione dell'attività nel contesto della ASD DOJO Trieste possa arrecare pregiudizio ai Tesserati e/o alle Tesserate, potrà disporsi la sospensione cautelare dalle attività sportive in attesa della definizione del procedimento endoassociativo.
		\item Dell'avvio del procedimento di cui al comma 1 nonché dell'esito dello stesso dovrà essere data tempestiva notizia al Responsabile per le politiche di \textit{Safeguarding} del sodalizio e al Responsabile per le politiche di \textit{Safeguarding} istituito presso la Federazione.
		\item I componenti degli organi e degli uffici della ASD DOJO Trieste coinvolti nell'espletamento delle procedure di cui al presente articolo assumono l'onere di riservatezza.
		\item Restano salve le azioni e i provvedimenti del Responsabile per le politiche di \textit{Safeguarding} istituito presso la Federazione, della Procura Federale e degli Organi di Giustizia Federali.
	\end{enumerate}
	
	\section{Rinvio}
	\begin{enumerate}
		\item Per quanto non previsto nel presente Codice di condotta si rinvia a tutte le disposizioni vigenti in materia.
	\end{enumerate}
	
	\section{Entrata in vigore e modifiche}
	\begin{enumerate}
		\item Il presente Codice, approvato a norma dello Statuto della ASD DOJO Trieste viene trasmesso al Responsabile per le politiche di \textit{Safeguarding} istituito presso la Federazione (\textit{Safeguarding Office}), per l'attività di vigilanza che gli è propria.
		\item Le modifiche al presente codice, anche se apportate su indicazione della Federazione, devono essere adottate a norma del primo comma del presente articolo.
	\end{enumerate}

\end{document}