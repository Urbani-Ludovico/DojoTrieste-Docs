% !TeX spellcheck = it_IT
% !TeX encoding = UTF-8
\documentclass{djtsdoc}

\title{{\normalfont Statuto della}\\[5pt] A.S.D. DOJO Trieste}
\date{24 novembre 2023}

\usepackage{titlesec}
\titleformat{\part}[display]{\centering\Large\bfseries}{Titolo \thepart}{0pt}{\large}
\titlespacing*{\part}{0pt}{0.4in}{0.3in}
\titleformat{\section}{\normalfont\large\bfseries}{Articolo \thesection\ -\ }{0pt}{}
\titlespacing*{\section}{0pt}{0.4in}{0.2in}

\begin{document}
	\maketitle
	
	\part{Denominazione, sede, oggetto e durata}
	\section{Denominazione}
	\begin{enumerate}
		\item È costituita, ai sensi e per gli effetti di quanto contenuto nel primo libro del codice civile e nel D.Lgs. 36/2021 e ss. mod., un'Associazione sportiva dilettantistica denominata \textbf{"Associazione sportiva dilettantistica DOJO Trieste"}, in breve \textbf{"A.S.D. DOJO Trieste"} (d'ora in poi "Associazione"), attualmente senza personalità giuridica che si riserva di chiederla con delibera di Assemblea ordinaria ai sensi dell'articolo 14, D.Lgs. 39/2021.
		\item L'Associazione ha la sede legale in Via Giuseppe Sinico n. 1 nel comune di Trieste in provincia di Trieste. Eventuali variazioni di sede purché nello stesso Comune possono essere approvate con deliberazione del Consiglio Direttivo e non comportano modifica statutaria.
		\item Potranno essere istituite sedi secondarie, succursali o uffici sia amministrativi che di rappresentanza, sia in Italia che all'estero.
		\item Nella denominazione, negli atti e nella corrispondenza è obbligatorio l'uso della locuzione "Associazione sportiva dilettantistica", anche in acronimo ASD.
		\item L'Associazione sportiva dilettantistica si impegna a trasmettere, in via telematica, entro il 31 gennaio dell'anno successivo, una dichiarazione all'ente affiliante riguardante l'aggiornamento dei dati ai sensi dell'articolo 6.3, D.Lgs. 39/2021, l'aggiornamento degli amministratori in carica e ogni altra modifica intervenuta nell'anno precedente.
	\end{enumerate}
	
	\section{Oggetto}
	\begin{enumerate}
		\item L'Associazione è apolitica e non ha scopo di lucro.
		\item Durante la vita dell'Associazione non potranno essere distribuiti, anche in modo indiretto o differito, avanzi di gestione, nonché fondi, riserve o capitale.
		\item L'Associazione è altresì caratterizzata dalla democraticità della struttura, dall'uguaglianza dei diritti di tutti gli associati, dall'elettività delle cariche associative.
		\item L'Associazione, riconosciuta ai fini sportivi ai sensi dell'articolo 10, D.Lgs. 36/2021 e iscritta al Registro Nazionale delle Attività Sportive Dilettantistiche  tenuto dal dipartimento dello Sport ai sensi delle disposizioni di legge vigenti esercita in via stabile e principale l'organizzazione e la gestione di attività sportivo dilettantistica ai sensi dell'articolo 7.1, lettera b), D.Lgs. 36/2021 ivi compresa la formazione, la didattica, la preparazione e l'assistenza all'attività sportiva dilettantistica. Nello specifico ha per finalità lo sviluppo e la diffusione di attività sportive dilettantistiche connesse alla disciplina del JUDO  e alle arti marziali, e più in generale delle discipline sportive considerate ammissibili dai regolamenti e dalle disposizioni del Coni e del Registro delle Attività Sportive tenuto dal Dipartimento Sport della Presidenza del Consiglio dei Ministri, intese come mezzo di formazione psico-fisica e morale degli associati, mediante la gestione di ogni forma di attività idonea a promuovere la conoscenza e la pratica della detta disciplina.
		\item Per il miglior raggiungimento degli scopi sociali, l'Associazione potrà, tra l'altro, svolgere, prevalentemente in favore dei propri associati, l'attività didattica per l'avvio, l'aggiornamento e il perfezionamento nello svolgimento della pratica sportiva delle discipline sopra indicate.
		\item Nei limiti previsti dall'articolo 9, D.Lgs. 36/2021 e dalla normativa di attuazione, è facoltà dell'Associazione svolgere attività secondaria e strumentale, purché strettamente connessa al fine istituzionale e nei limiti ivi indicati quali a mero titolo esemplificativo:
		\begin{itemize}
			\item attività ricreativa in favore dei propri soci, ivi compresa, se del caso, la gestione di un posto di ristoro;
			\item la gestione di centri benessere o fisioterapici;
			\item la vendita di articoli sportivi;
			\item la promozione di attività sportiva, ricreativa, culturale e, in generale, l'attività svolta da associati o tesserati alle organizzazioni sportive di riferimento anche attraverso la partecipazione a manifestazioni fieristiche, lo svolgimento di azioni pubblicitarie, l'espletamento di studi e ricerche di mercato, la predisposizione di cataloghi e qualsiasi altro mezzo di promozione ritenuto idoneo.
		\end{itemize}
		\item L'Associazione garantirà la partecipazione dei propri atleti e dei propri tecnici alle assemblee federali per consentire loro l'elezione dei propri rappresentanti in Consiglio federale.
		\item L'Associazione potrà collaborare o anche aderire ad altri enti, sia pubblici sia privati, ivi compresi enti scolastici, con finalità similari, affini o complementari con i quali siano condivisi gli scopi e gli intendimenti.
		\item L'Associazione accetta incondizionatamente di conformarsi allo Statuto, alle norme e alle direttive del Comitato Internazionale Olimpico, (C.I.O.), del Comitato Olimpico Nazionale Italiano (C.O.N.I.), del Comitato Italiano Paralimpico (C.I.P.), nonché agli statuti e regolamenti sia vigenti che a quelli che fossero emanati successivamente all'approvazione del presenta Statuto e alle disposizioni della  Federazioni Italiana Judo Lotta Karate Arti Marziali (F.I.J.L.K.A.M.), nonché alle disposizioni delle ulteriori delle Federazioni Sportive Nazionali,  delle Discipline Sportive Associate, degli Enti di Promozione Sportiva riconosciuti dal Coni e dal CIP, a cui l'Associazione vorrà affiliarsi. L'Associazione si impegna altresì a rispettare le disposizioni emanate dalle federazioni internazionali di riferimento in merito all'attività sportiva praticata. L'Associazione si impegna pertanto ad accettare eventuali provvedimenti disciplinari, che gli organi competenti dell'organismo affiliante  dovessero adottare a suo carico, nonché le decisioni che le autorità sportive dovessero prendere in tutte le vertenze di carattere associativo, tecnico e disciplinare attinenti alla vita della Associazione sportiva.
		\item L'Associazione si impegna inoltre a garantire l'attuazione ed il pieno rispetto dei provvedimenti del CONI e/o delle Federazioni, Enti di Promozione Sportiva o discipline sportive associate, e in generale di tutte le disposizioni emanate a presidio della lotta alla violenza di genere ai sensi dell'articolo 16, D.Lgs. 39/2021
		\item L'Associazione inoltre si impegna ad applicare tutte le disposizioni di cui all'art. 33 del D.Lgs 36/2021  in materia del lavoratori sportivi e dei minori.
	\end{enumerate}
	
	\section{Durata}
	\begin{enumerate}
		\item L'Associazione ha durata illimitata e potrà essere sciolta soltanto con delibera dell'Assemblea straordinaria degli associati.
	\end{enumerate}
	
	\part{Della vita associativa}
	\section{Domanda di ammissione}
	\begin{enumerate}
		\item Possono far parte dell'Associazione in qualità di soci le persone fisiche che ne facciano richiesta e che siano dotate di una irreprensibile condotta morale, civile e sportiva.
		\item Ai fini sportivi, per "irreprensibile condotta" deve intendersi a titolo esemplificativo e non limitativo una condotta conforme ai principi della lealtà, della probità e della rettitudine sportiva in ogni rapporto collegato all'attività sportiva, con l'obbligo di astenersi da ogni forma d'illecito sportivo e da qualsivoglia indebita esternazione pubblica lesiva della dignità, del decoro e del prestigio dell'Associazione,  della \mbox{FIJLKAM}, delle Federazioni Sportive Nazionali, Discipline Sportive Associate, Enti di Promozione Sportiva a cui sarà affiliata, oltre che delle competenti autorità sportive.
		\item Viene espressamente escluso ogni limite sia temporale che operativo al rapporto associativo e ai diritti che ne derivano, fermo restando il diritto di recesso.
		\item Chi intenda aderire all'Associazione deve presentare domanda scritta su apposito modulo predisposto dal  Consiglio Direttivo recante, oltre ai dati anagrafici e un indirizzo di posta elettronica  in corso di validità per la trasmissione delle comunicazioni, anche la dichiarazione di conoscere e condividere le finalità dell'Associazione e l'impegno di rispettare le prescrizioni contenute nello Statuto, nel  regolamento e nelle deliberazioni degli Organi Sociali.  Il Consiglio Direttivo delibera in merito alla domanda entro 60 (sessanta) giorni dal ricevimento della stessa: in caso di accoglimento l'ammissione all'Associazione  si perfeziona con il pagamento da parte del nuovo socio della quota associativa annuale e degli eventuali ulteriori contributi o quote previsti dal Regolamento associativo.   Il Consiglio Direttivo può respingere la domanda con delibera motivata che deve essere  tempestivamente comunicata al richiedente per posta elettronica certificata  o lettera raccomandata con avviso di ricevimento. Avverso il rigetto l'interessato può proporre reclamo  all'Assemblea generale, a pena di decadenza, entro e non oltre 15  (quindici) giorni dal ricevimento del diniego. La discussione sull'ammissione avviene alla prima Assemblea utile successiva al ricevimento del ricorso.
		\item La deliberazione di ammissione del nuovo socio è senza indugio annotata nel libro degli associati tenuto a cura del Consiglio Direttivo.
		\item L'associato può anche essere tesserato, per il tramite dell'Associazione, alla Federazione Sportiva Nazionale e/o agli Enti di Promozione Sportiva a cui l'Associazione aderisce.
		\item Le quote associative sono personali, non sono trasferibili, rivalutabili né restituibili agli associati, eredi e/o aventi causa.
		\item In caso di domanda di ammissione a Socio presentate da minorenni le stesse dovranno essere controfirmate dall'esercente la responsabilità genitoriale, con indicazione di eventuali altri soggetti che la esercitino. Colui che sottoscrive la domanda rappresenta il minore a tutti gli effetti nei confronti dell'Associazione e risponde verso la stessa per tutte le obbligazioni del minorenne, salvo subentro o sostituzione di detto soggetto, da comunicarsi tempestivamente all'Associazione.
		\item L'Assemblea può deliberare che, all'atto della prima domanda di ammissione a socio, debba essere versata, oltre la quota associativa prevista per l'esercizio in cui è stata presentata la domanda, anche una quota di ingresso secondo un ammontare predeterminato dalla stessa Assemblea.
		\item Con la sottoscrizione della domanda di ammissione il socio accetta che i propri dati personali siano comunicati agli organismi che procedono al riconoscimento ai fini sportivi e alla relativa certificazione della attività sportiva dilettantistica svolta.
		\item Per tutte le procedure non specificatamente indicate in questa sede, si demanda al Regolamento Sociale.
	\end{enumerate}
	
	\section{Diritti e doveri dei soci}
	\begin{enumerate}
		\item Tutti i soci sono effettivi e hanno i medesimi diritti, senza discriminazione alcuna, che esercitano nel rispetto delle norme statutarie e regolamentari.
		\item In particolare, i soci hanno:
		\begin{enumerate}
			\item il diritto a partecipare alle attività associative;
			\item il diritto di voto per l'approvazione delle modificazioni dello Statuto e dei regolamenti e per la nomina degli organi sociali dell'Associazione;
			\item il diritto di voto per l'approvazione del bilancio consuntivo di esercizio annuale;
			\item il diritto di candidarsi, se maggiorenni, alle cariche sociali;
			\item il diritto di esaminare i libri sociali facendone richiesta motivata al Consiglio Direttivo, che stabilisce i tempi e le modalità di esercizio di tale diritto in maniera comunque tale da non renderne impossibile o eccessivamente oneroso per i soci il suo concreto esercizio.
		\end{enumerate}
		\item Il minore esercita il diritto di partecipazione nell'Assemblea mediante l'esercente la responsabilità genitoriale individuato ai sensi del precedente articolo 4.8.
		\item Il diritto all'elettorato passivo verrà automaticamente acquisito dal socio minorenne alla prima Assemblea utile svoltasi dopo il raggiungimento della maggiore età.
		\item I soci sono tenuti al puntuale pagamento delle quote associative e dei contributi deliberati dal Consiglio Direttivo e dall'Assemblea, nonché al rispetto delle norme statutarie e regolamentari dell'Associazione e delle disposizioni emanate dal Consiglio Direttivo.
	\end{enumerate}
	
	\section{Decadenza dei soci}
	\begin{enumerate}
		\item La qualifica di socio si perde per dimissioni,  esclusione o causa di morte.
		\item Le dimissioni da socio devono essere presentate per iscritto, a mezzo posta elettronica o raccomandata a/r al Consiglio Direttivo entro il termine dell'esercizio sociale: il socio dimissionario è tenuto al pagamento della quota  associativa riferita all'esercizio  sociale nel corso del quale ha cessato la propria appartenenza all'Associazione.
		\item Gli associati sono automaticamente esclusi  dall'Associazione in caso di morosità protrattasi per un periodo di oltre tre mesi  decorrenti dall'inizio dell'esercizio sociale.
		\item Nel caso di gravi violazioni delle regole associative e dei principi e valori fondativi dell'Associazione l'associato può essere escluso con deliberazione motivata del Consiglio Direttivo, comunicata allo interessato a mezzo di posta elettronica certificata o raccomandata a/r. Avverso l'esclusione  l'interessato può proporre reclamo  all'Assemblea generale, a pena di decadenza, entro e non oltre 15 (quindici) giorni dal ricevimento del provvedimento. La discussione sull'esclusione avviene alla prima Assemblea utile successiva al ricevimento del ricorso. Il provvedimento di esclusione rimane sospeso fino alla decisione dell'Assemblea. L'esclusione diventa operativa con l'annotazione del provvedimento nel libro dei soci che avviene decorsi  30 (trenta) giorni dal ricevimento del provvedimento ovvero a seguito di delibera dell'Assemblea cha abbia ratificato il provvedimento di esclusione adottato dal Consiglio Direttivo.
		\item La perdita per qualsiasi causa della qualifica di associato non attribuisce a quest'ultimo, agli eredi e/o aventi causa alcun diritto alla restituzione delle quote e dei contributi versati in qualsiasi tempo all'Associazione.
	\end{enumerate}
	
	\part{Degli organi associativi}
	\section{Organi sociali}
	\begin{enumerate}
		\item L'ordinamento interno dell'Associazione si basa sui principi di democrazia e di uguaglianza dei diritti di tutti gli associati. Le cariche sociali sono elettive.
		\item Sono organi dell'Associazione:
		\begin{enumerate}
			\item l'Assemblea generale degli associati;
			\item il Consiglio Direttivo;
			\item il Presidente;
			\item il Vice Presidente;
			\item il Segretario;
			\item il Tesoriere;
			\item il Collegio dei Revisori dei Conti o il Revisore dei Conti, qualora istituito.
		\end{enumerate}
		\item I requisiti per ricoprire cariche sociali sono:
		\begin{enumerate}
			\item aver raggiunto la maggiore età;
			\item aver acquisito la qualità di socio al momento dello svolgimento dell'assemblea;
			\item essere in regola con il versamento delle quote sociali;
			\item non ricoprire qualsiasi carica sociale in altre società o associazioni sportive dilettantistiche nell'ambito della medesima Federazione Sportiva Nazionale, Discipline Sportive Associate o Enti di Promozione Sportiva riconosciuti dal CONI, ovvero nell'ambito della medesima disciplina facente capo a un ente di promozione sportiva come previsto dall'art. 11, D.lgs. 28 febbraio 2021 n. 36, e ss. mm. e ii.
			\item non aver riportato condanne passate in giudicato per reati non colposi a pene detentive superiori ad un anno ovvero a pene che comportino l'interdizione dai pubblici uffici superiore ad un anno;
			\item non aver riportato nell'ultimo decennio, salvo riabilitazione, squalifiche od inibizioni sportive definitive complessivamente superiori a un anno da parte del CONI, delle Federazioni Sportive Nazionali, delle Discipline Sportive Associate, degli Enti di Promozione Sportiva o di organismi sportivi internazionali riconosciuti;
			\item non aver subito sanzioni di sospensione dall'attività sportiva a seguito di utilizzo di sostanze o di metodi che alterano le naturali prestazioni fisiche.
		\end{enumerate}
		\item La mancanza dei requisiti di cui al precedente comma accertata o verificatasi dopo l'elezione comporta la decadenza dalla carica.
		\item Tutte le cariche sociali sono e vengono assunte a titolo gratuito, fermo restando il rimborso delle  spese sostenute e documentate in esecuzione della carica. Le cariche non sono cumulabili all'interno dell'Associazione.
	\end{enumerate}
	
	\section{Assemblea generale dei soci. Convocazione e funzionamento}
	\begin{enumerate}
		\item L'Assemblea generale dei soci è il massimo organo deliberativo dell'Associazione.
		\item L'Assemblea è composta da tutti gli associati iscritti nel libro degli associati e in regola con il versamento delle quote associative.
		\item L'Assemblea è indetta dal Consiglio Direttivo e convocata dal Presidente dell'Associazione o, in caso di suo impedimento, dal Vice  Presidente oppure, in subordine, dal Consigliere più anziano di carica sia in sessione ordinaria che straordinaria.
		\item L'Assemblea deve essere convocata almeno una volta all'anno, entro quattro mesi dalla chiusura dell'esercizio sociale, per l'approvazione del bilancio consuntivo e per l'esame del bilancio preventivo. Fino al momento dell'approvazione del rendiconto preventivo il Consiglio Direttivo è autorizzato all'esercizio provvisorio sulla base del preventivo approvato l'anno precedente, suddiviso in dodicesimi.
		\item La convocazione dell'Assemblea straordinaria può essere richiesta al Consiglio Direttivo da:
		\begin{enumerate}
			\item almeno la metà più 1 (uno)  degli associati in regola con il pagamento delle quote associative e non sottoposti a provvedimenti disciplinari in corso di esecuzione che ne propongono l'ordine del giorno;
			\item almeno la metà più 1 (uno) dei componenti il Consiglio Direttivo.
		\end{enumerate}
		\item L'Assemblea deve essere convocata presso la sede dell'Associazione o in altro luogo idoneo a garantire la massima partecipazione degli associati  purché nel  medesimo comune.
		\item Sono ammesse le audio/video assemblee ai sensi dell'articolo 13 del presente Statuto.
		\item L'Assemblea, sia ordinaria che straordinaria, viene convocata mediante pubblicazione sul sito istituzionale di apposito "Avviso di convocazione", da comunicare altresì all'indirizzo di posta elettronica indicato in sede di adesione da ogni associato, con almeno 8 giorni di anticipo rispetto alla data della riunione. In mancanza di sito Internet la pubblicazione dovrà avvenire sulla bacheca sociale entro il medesimo termine.
		\item L'avviso di convocazione contiene data e ora della riunione, il luogo, l'ordine del giorno sia in prima che in seconda convocazione. L'Assemblea in seconda convocazione deve svolgersi almeno un'ora dopo la prima convocazione.
		\item L'Assemblea, quando è regolarmente convocata e costituita, rappresenta l'universalità degli associati e le deliberazioni da essa legittimamente adottate obbligano tutti gli associati, anche se non intervenuti o dissenzienti.
		\item L'Assemblea è presieduta dal Presidente del Consiglio Direttivo o, in caso di suo impedimento, dal Vice Presidente oppure, in subordine, dal Consigliere più anziano ovvero, in ultima istanza, dalla persona di volta in volta designata dagli intervenuti.
		\item Il Presidente dirige e regola le discussioni e stabilisce le modalità e l'ordine delle votazioni.
		\item L'Associazione tiene, a cura del Consiglio Direttivo, un libro delle adunanze e delle deliberazioni dell'Assemblea, in cui devono essere trascritti anche i verbali redatti per atto pubblico.
		\item L'Assemblea nomina un segretario e, se necessario, uno o più scrutatori.
		\item Di ogni Assemblea si dovrà redigere apposito verbale firmato dal Presidente della stessa, dal segretario e, se nominati, dagli scrutatori. Copia dello stesso deve essere messo a disposizione di tutti gli associati con le formalità ritenute più idonee dal Consiglio Direttivo a garantirne la massima diffusione.
		\item Laddove l'Assemblea abbia carattere elettivo delle cariche sociali o comporti la modifica del presente Statuto, una copia del verbale va inviata anche agli organismi sportivi a cui l'Associazione è affiliata.
		\item L'assistenza del segretario non è necessaria quando il verbale dell'Assemblea sia redatto da un notaio.
		\item L'Assemblea delibera sui punti contenuti nell'ordine del giorno.
		\item Proposte o mozioni di qualsiasi natura che si intendano presentare all'Assemblea devono essere scritte e sotto firmate da almeno 10 (dieci) soci e presentate al Presidente almeno 10 (dieci) giorni prima della data fissata per l'adunanza.
		\item Le mozioni urgenti e le proposte di modifica dell'ordine del giorno in merito alla successione degli argomenti da trattare possono essere presentate, anche a voce, durante i lavori dell'Assemblea e possono essere inserite nell'ordine del giorno con il voto favorevole della maggioranza dei presenti.
	\end{enumerate}
	
	\section{Partecipazione all'assemblea}
	\begin{enumerate}
		\item Potranno prendere parte alle assemblee ordinarie e straordinarie dell'Associazione i soli associati in regola con il pagamento delle quote associative e non soggetti a provvedimenti disciplinari in corso di esecuzione.
		\item Il diritto di voto è esercitato dagli associati maggiorenni e, per gli associati minorenni, dai soggetti indicati all'articolo 4.8.
		\item Ogni socio ha diritto a un voto e può rappresentare in Assemblea, per mezzo di delega scritta, un altro associato.
	\end{enumerate}

	\section{Assemblea ordinaria}
	\begin{enumerate}
		\item L'assemblea dei soci:
		\begin{enumerate}
			\item approva il rendiconto economico e finanziario;
			\item elegge  e revoca il Presidente e i componenti del Consiglio Direttivo;
			\item determina gli indirizzi secondo i quali deve svolgersi l'attività dell'Associazione e delibera sulle proposte di adozione e modifica di eventuali regolamenti;
			\item elegge e revoca, i componenti  del Collegio dei Revisori dei Conti;
			\item delibera sulla responsabilità dei componenti degli organi sociali e promuove azione di responsabilità nei loro confronti;
			\item delibera sul diniego di ammissione del socio o sulle delibere di esclusione eventualmente impugnate;
			\item individua le attività diverse da quelle di interesse generale che, nei limiti consentiti dalla legge, possono essere svolte dall'Associazione;
			\item delibera in merito l'approvazione dei regolamenti sociali ivi compresi i modelli organizzativi di cui al comma 2, articolo 16, D.Lgs. 36/2021;
			\item delibera sull'ordine del giorno, mozioni e ogni altra materia a essa riservata dalla legge o dal presente Statuto.
		\end{enumerate}
	\end{enumerate}
	
	\section{Assemblea straordinaria}
	\begin{enumerate}
		\item L'assemblea straordinaria delibera:
		\begin{enumerate}
			\item sull'approvazione e sulle proposte di modifica dello Statuto;
			\item sulla trasformazione, anche ai sensi dell'articolo 24 dello Statuto, la fusione e lo scioglimento dell'Associazione e sulla devoluzione del suo patrimonio;
			\item sui diritti reali immobiliari;
			\item sulla elezione del Consiglio Direttivo decaduto;
			\item sugli altri argomenti posti all'ordine del giorno attinenti atti di straordinaria amministrazione.
		\end{enumerate}
	\end{enumerate}
	
	\section{Validità assembleare}
	\begin{enumerate}
		\item L'Assemblea ordinaria è validamente costituita in prima convocazione con la presenza della maggioranza assoluta degli associati aventi diritto di voto e delibera validamente con voto della maggioranza dei presenti.
		\item L'Assemblea straordinaria è validamente costituita in prima convocazione quando sono presenti 2/3 degli associati aventi diritto di voto e delibera con il voto favorevole della maggioranza dei presenti.
		\item Dall'ora successiva la prima convocazione, sia l'Assemblea ordinaria che l'Assemblea straordinaria, compresa l'Assemblea straordinaria per la modifica dell'atto costitutivo e dello Statuto,  sono validamente costituite qualunque sia il numero degli associati intervenuti e delibera con il voto favorevole della maggioranza dei presenti.
		\item Per deliberare lo scioglimento dell'Associazione e la devoluzione del patrimonio occorre il voto favorevole di almeno i 3/4 degli associati ai sensi dell'articolo 21, comma 3 c.c.
	\end{enumerate}
	
	\section{Audio/video assemblee}
	\begin{enumerate}
		\item È possibile tenere le riunioni dell'Assemblea, con interventi dislocati in più luoghi, audio/video collegati, e ciò alle condizioni previste dalla legge, cui dovrà essere dato atto nei relativi verbali.
		\item È in ogni caso necessario che:
		\begin{itemize}
			\item comunque debbono essere presenti nel medesimo luogo il Presidente e il segretario della riunione;
			\item vi sia la possibilità, per il Presidente, di identificare i partecipanti, di regolare lo svolgimento assembleare e di constatare e proclamare i risultati delle votazioni;
			\item venga garantita la possibilità di tenere il verbale completo della riunione;
			\item venga garantita la discussione in tempo reale delle questioni, lo scambio di opinioni, la possibilità di intervento e la possibilità di visionare i documenti, da depositarsi presso la sede nei giorni precedenti l'adunanza;
			\item sia garantita la possibilità di partecipare alle votazioni;
			\item sia consentito agli intervenuti di partecipare in tempo reale alla discussione e in maniera simultanea alla votazione sugli argomenti posti all'ordine del giorno nonché di trasmettere, ricevere e visionare documenti;
			\item vengano indicati nell'avviso di convocazione i luoghi audio collegati o audio-video collegati - a cura della società a cui gli intervenuti possono collegarsi attestando la loro presenza.
		\end{itemize}
		In presenza dei suddetti presupposti, l'Assemblea si considera tenuta nel luogo in cui si trova il Presidente e dove deve pure trovarsi il segretario della riunione, onde consentire la stesura e la sottoscrizione del verbale sul relativo libro.
	\end{enumerate}
	
	\section{Il consiglio direttivo}
	\begin{enumerate}
		\item Il Consiglio Direttivo è l'organo responsabile della gestione dell'Associazione e cura collegialmente l'esercizio dell'attività associativa.
		\item Il Consiglio Direttivo è composto da 5 (cinque) membri eletti dall'Assemblea, ivi compreso il Presidente.
		\item Il Consiglio Direttivo, nel proprio ambito elegge, il Vice Presidente, il Segretario e il Tesoriere; queste ultime due cariche possono essere ricoperte anche dalla stessa persona.
		\item I Consiglieri eletti devono riunirsi entro 15 (quindici) giorni dalla avvenuta Assemblea elettiva su convocazione del Presidente uscente o, in caso di mancata convocazione da parte dello stesso, su richiesta scritta della maggioranza del Consiglio Direttivo uscente.
		\item La presenza alla prima riunione dell'associato eletto costituisce formale accettazione della nomina. Gli assenti ingiustificati sono da ritenersi dimissionari.
		\item Il Consiglio Direttivo dura in carica 4 (quattro) anni e i suoi componenti sono rieleggibili.
		\item La rappresentanza legale dell'Associazione spetta istituzionalmente al Presidente del Consiglio Direttivo, che cura l'esecuzione delle delibere  dell'Assemblea e del Consiglio Direttivo, e, per compiti specifici, agli altri Consiglieri designati dal Consiglio Direttivo sulla base di apposita deliberazione.
		\item Il Presidente può, in caso di urgenza, esercitare i poteri del Consiglio Direttivo salvo ratifica da parte di quest'ultimo alla prima riunione utile.
		\item Il Consiglio Direttivo potrà avere luogo altresì "da remoto" ai sensi del precedente articolo 13 dello Statuto.
		\item Le riunioni sono valide se è presente la maggioranza assoluta dei componenti, e le deliberazioni sono approvate a maggioranza dei presenti.
		\item In caso di parità prevale il voto del Presidente o, in sua mancanza, del Vice Presidente.
		\item Il Consiglio Direttivo tiene, a sua cura, un libro delle proprie adunanze e deliberazioni.
		\item Le deliberazioni del Consiglio Direttivo devono risultare da un verbale sottoscritto da chi ha presieduto la riunione e dal Segretario.
		\item Il verbale deve essere messo a disposizione di tutti gli associati con le formalità ritenute più idonee dal Consiglio Direttivo atte a garantirne la massima diffusione.
	\end{enumerate}
	
	\section{Dimissioni e cause di decadenza del Consiglio Direttivo e del Presidente}
	\begin{enumerate}
		\item Il Consiglio Direttivo decade:
		\begin{enumerate}
			\item per dimissioni contemporanee della metà più 1 (uno) dei suoi componenti;
			\item per dimissioni o impedimento definitivo del Presidente;
			\item per contemporanea vacanza, per qualsivoglia causa, della metà più 1 dei suoi componenti;
			\item per mancata approvazione del bilancio consuntivo di esercizio da parte dell'Assemblea.
		\end{enumerate}
		\item In queste ipotesi il Presidente del Consiglio Direttivo o, in caso di suo impedimento o vacanza, il Vice Presidente oppure, in subordine, il Consigliere più anziano, dovrà provvedere entro 60 (sessanta) giorni alla convocazione dell'Assemblea, da celebrarsi nei successivi 30 (trenta) giorni, curando nel frattempo l'ordinaria amministrazione.
		\item Fino alla sua nuova costituzione e limitatamente agli affari urgenti e alla ordinaria amministrazione, le funzioni saranno svolte dal Presidente in regime di prorogatio.
		\item Nel caso in cui, per qualsiasi ragione, durante il corso dell'esercizio venissero a mancare contestualmente tanti Consiglieri che non superino la metà del Consiglio Direttivo, si procederà alla  integrazione del Consiglio con il subentro del primo candidato non eletto nella votazione alla carica di Consigliere. In assenza il  Consiglio proseguirà in numero ridotto fino alla prima Assemblea utile che provvederà alle votazioni per reintegrare i membri vacanti.
		\item Oltre che nei casi di decadenza del Consiglio Direttivo, il Presidente decade:
		\begin{enumerate}
			\item per dimissioni;
			\item per vacanza, a qualsivoglia causa dovuta.
		\end{enumerate}
		\item In queste ultime ipotesi, il Vice Presidente o, in subordine, il Consigliere più anziano, dovrà entro 60 (sessanta) giorni provvedere alla convocazione dell'Assemblea, da celebrarsi nei successivi 30 (trenta) giorni, curando nel frattempo l'ordinaria amministrazione.
		\item Fino alla sua nuova costituzione e limitatamente agli affari urgenti e alla ordinaria amministrazione, le funzioni saranno svolte dal Vice Presidente o dal Consigliere più anziano, in regime di \textit{prorogatio}.
	\end{enumerate}
	
	\section{Convocazione del consiglio direttivo}
	\begin{enumerate}
		\item Il Consiglio Direttivo si riunisce senza formalità tutte le volte che si rende necessario ma almeno 4 (quattro) volte l'anno su iniziativa del Presidente e straordinariamente se la maggioranza dei Consiglieri ne chiedono la convocazione.
	\end{enumerate}
	
	\section{Compiti del consiglio direttivo}
	\begin{enumerate}
		\item Il Consiglio Direttivo è dotato dei più ampi poteri per la gestione ordinaria dell'Associazione. A esso competono in particolare:
		\begin{enumerate}
			\item la redazione annuale e la presentazione in Assemblea, del bilancio consuntivo dell'attività svolta nel corso dell'anno solare precedente e di quello preventivo;
			\item indire le assemblee ordinarie dei soci da convocarsi almeno 1 volta all'anno, nonché le assemblee straordinarie anche nel rispetto del presente Statuto;
			\item determinare l'importo delle quote associative;
			\item assumere le decisioni inerenti spese ordinarie di esercizio e in c/capitale, per la gestione dell'Associazione;
			\item assumere le decisioni relative alle attività e ai servizi istituzionali, complementari e commerciali da intraprendere per il migliore conseguimento delle finalità istituzionali dell'Associazione;
			\item assumere le decisioni inerenti la direzione del personale dipendente e coordinamento dei collaboratori e dei professionisti di cui si avvale l'Associazione nonché di eventuali volontari e curare l'esecuzione degli adempimenti di cui al D.Lgs. 36/2021 in materia di lavoro sportivo;
			\item la presentazione di un piano programmatico relativo alle attività da svolgere nel nuovo anno sociale;
			\item l'elaborazione di proposte di modifica dello Statuto, o di emanazione e modifica dei regolamenti sociali;
			\item l'istituzione di commissioni e la nomina di rappresentanti in organismi pubblici e privati, federazioni e altri enti;
			\item la facoltà di nominare tra gli associati, soggetti esterni all'ambito consigliare, delegati a svolgere particolari funzioni stabilite di volta in volta dal Consiglio Direttivo stesso;
			\item redigere gli eventuali regolamenti interni relativi all'attività sociale da sottoporre all'approvazione dell'Assemblea degli associati;
			\item adottare provvedimenti disciplinari nei confronti dei soci, i quali potranno impugnarli dinanzi all'Assemblea come da separato Regolamento da  adottare  entro la prima assemblea successiva alla approvazione del presente Statuto;
			\item delibera sulle domande di ammissione degli associati o su eventuali cause di esclusione;
			\item nomina il responsabile della protezione dei minori di cui all'articolo 33, comma 6, D.Lgs. 36/2021;
			\item qualsiasi altra funzione espressamente prevista nel presente Statuto o che non sia espressamente attribuita agli altri Organi.
		\end{enumerate}
	\end{enumerate}
	
	\section{Il presidente}
	\begin{enumerate}
		\item Il Presidente è eletto dall'Assemblea con la maggioranza dei voti dei presenti/rappresentati.
		\item Dura in carica 4 (quattro) anni ed è rieleggibile.
		\item Egli presiede l'Assemblea ed il Consiglio Direttivo e ne provvede alla convocazione, vigila sulla corretta esecuzione delle delibere di tutti gli organi sociali dei quali controlla il funzionamento e il rispetto della competenza.
		\item Ha la rappresentanza legale dell'Associazione.
		\item Nei casi di urgenza il Presidente può esercitare i poteri del Consiglio, salvo ratifica da parte di questo alla prima riunione utile successiva, da tenersi comunque entro 30 giorni dalla decisione.
	\end{enumerate}
	
	\section{Il vice presidente}
	\begin{enumerate}
		\item Il Vice Presidente viene eletto nel proprio ambito dal Consiglio Direttivo a maggioranza dei presenti/rappresentati e sostituisce il Presidente in caso di sua assenza o impedimento temporaneo e in quelle mansioni per le quali venga espressamente delegato.
	\end{enumerate}
	
	\section{Il segretario e il tesoriere}
	\begin{enumerate}
		\item Le funzioni di Segretario e Tesoriere possono essere conferite anche alla stessa persona.
		\item Qualora esse siano attribuite a persone diverse, in caso di impedimento del Tesoriere a svolgere le proprie funzioni, ovvero nell'ipotesi di dimissioni o di revoca del medesimo, le funzioni di questo sono assunte, per il tempo necessario a rimuovere le cause di impedimento, ovvero a procedere a nuova nomina, dal Segretario o dal Vice Presidente.
		\item Il Segretario, temporaneamente impedito, ovvero dimissionario o revocato, è sostituito con le stesse modalità dal Tesoriere o dal Vice Presidente.
		\item Il Segretario redige i verbali delle riunioni degli organi sociali e ne cura la trascrizione nei relativi libri e registri; dà esecuzioni alle deliberazioni del Presidente e del Consiglio Direttivo, segue le procedure di tesseramento dei soci,  attende alla corrispondenza, predispone la modulistica inerente  le attività principali e secondarie dell'Associazione anche avvalendosi dell'ausilio di soggetti tecnici esterni incaricati dal Consiglio Direttivo.
		\item Il Tesoriere provvede, anche in collaborazione con gli altri membri del Consiglio Direttivo ed avvalendosi altresì dell'ausilio di soggetti tecnici esterni incaricati dal Consiglio Direttivo stesso, alla tenuta delle scritture contabili, alla verifica del corretto svolgimento degli adempimenti fiscali, lavorativi e contributivi dell'Associazione. Egli inoltre sovrintende alla predisposizione del rendiconto annuale in termini economici e finanziari, e alla stesura del bilancio preventivo dell'esercizio successivo.
		\item Il Tesoriere è incaricato di compiere le operazioni formali di incasso di quote sociali nonché di quote di partecipazione ad eventi ed attività principali e secondarie, e di pagamento delle spese deliberate dal Consiglio Direttivo.
		\item Al Tesoriere spetta anche la funzione del controllo delle risultanze dei conti finanziari di cassa, banca, crediti e debiti.
	\end{enumerate}
	
	\section{Organo di revisione}
	\begin{enumerate}
			\item L'organo di revisione può essere eletto dall'Assemblea preferibilmente fra soggetti scelti fra le categorie di cui all'art. 2397 comma 2 c.c. Può essere sia monocratico che collegiale e resta in carica quattro   anni. I componenti possono essere rieletti per un massimo di tre mandati consecutivi.
			\item Controlla l'amministrazione dell'Associazione, la corrispondenza, il bilancio, le scritture contabili e vigila sul rispetto dello Statuto nonché sulla corretta istituzione   e conduzione dei rapporti di lavoro, sull'osservanza della legge e  sui principi di corretta amministrazione  anche con riferimento alle disposizioni e ai compiti allo stesso assegnati ex D.Lgs. 08.06.2021 n. 231 e ss.mm. e ii.
			\item Partecipa con un unico membro alle riunioni del Consiglio Direttivo e alle Assemblee, senza diritto di voto. Inoltre senza diritto di voto partecipa con tutti i propri membri alle Assemblee ove presenta la propria relazione annuale in tema di bilancio consuntivo ovvero controfirma il bilancio presentato dal Consiglio Direttivo.
			\item Tale organo si riunisce ogni 90 (novanta) giorni per le dovute verifiche contabili e amministrative, nonché qualora opportuno, previa convocazione del Presidente.
			\item Le adunanze e le decisioni devono essere riportate in un apposito verbale sottoscritto da tutti i componenti presenti.
			\item Per quanto compatibile con il presente Statuto si applicano le norme di cui agli articoli 2397 e ss., cod. civ.
	\end{enumerate}
	
	\part{Patrimonio e scritture contabili}
	\section{Il rendiconto economico}
	\begin{enumerate}
		\item L'anno sociale e l'esercizio finanziario iniziano il 1 gennaio e terminano i 31 dicembre di ogni anno.
		\item La redazione e la regolare tenuta del rendiconto economico-finanziario è obbligatoria.
		\item Il Consiglio Direttivo redige il bilancio dell'Associazione, sia preventivo che consuntivo, da sottoporre all'approvazione dell'Assemblea entro quattro mesi dalla chiusura dell'esercizio secondo le disposizioni del presente Statuto.
		\item Il bilancio consuntivo deve informare circa la complessiva situazione economico- finanziaria dell'Associazione.
		\item Il bilancio deve essere redatto con chiarezza e deve rappresentare in modo veritiero e corretto la situazione patrimoniale ed economico-finanziaria dell'Associazione, nel rispetto del principio della trasparenza nei confronti degli associati. In occasione della convocazione dell'Assemblea ordinaria, che riporta all'ordine del giorno l'approvazione del bilancio, deve essere messa a disposizione di tutti gli associati copia del bilancio stesso.
		\item L'intero Consiglio Direttivo, compreso il Presidente, decade in caso di mancata approvazione del bilancio da parte dell'Assemblea. In questo caso troverà applicazione quanto disposto dall'articolo 15.
	\end{enumerate}
	
	\section{Risorse economiche - fondo comune}
	\begin{enumerate}
		\item L'Associazione trae le risorse economiche per il suo funzionamento e per lo svolgimento delle sue attività da:
		\begin{itemize}
			\item quote di ammissione, quote associative, contributi e corrispettivi specifici versati dai soci per le attività svolte in diretta attuazione dei fini istituzionali;
			\item quote di iscrizione e di tesseramento, contributi e corrispettivi specifici versati dai tesserati per le attività svolte in diretta attuazione dei fini istituzionali;
			\item donazioni, eredità, legati e lasciti testamentari;
			\item erogazioni liberali da parte di persone fisiche, società, enti pubblici e privati;
			\item entrate derivanti da attività secondarie e strumentali agli scopi istituzionali;
			\item entrate derivanti da raccolte pubbliche di fondi e altre attività occasionali e saltuarie;
			\item entrate derivanti dall'organizzazione di gare o manifestazioni di carattere sportivo;
			\item ogni altra entrata che contribuisca al reperimento dei fondi necessari al raggiungimento degli scopi istituzionali, nel rispetto dei limiti e delle condizioni imposte dalla normativa vigente.
		\end{itemize}
		\item Il fondo comune, costituito – a titolo esemplificativo e non esaustivo – da avanzi di gestione, fondi, riserve e tutti i beni mobili e immobili  acquisiti a qualsiasi titolo dall'Associazione, non è mai ripartibile tra i soci durante la vita dell'Associazione né all'atto del suo scioglimento e non può essere destinato ad altri usi se non quelli per i quali l'associazione è costituita, ai sensi e per gli effetti dell'art. 8 D.Lgs n. 36/2021.
		\item È sempre vietata la distribuzione, anche indiretta, di utili e avanzi di gestione, fondi e riserve comunque denominati, ad associati, lavoratori e collaboratori, amministratori e altri componenti degli organi sociali, anche nel caso di recesso o di qualsiasi altra ipotesi di scioglimento individuale del rapporto, ai sensi e per gli effetti dell'art. 8 D.Lgs n. 36/2021.
		\item L'amministrazione di detti fondi e la gestione di tutti i rapporti giuridici attivi e passivi che ne conseguono, sono regolati dal Consiglio Direttivo.
		\item I versamenti dei soci non creano altri diritti di partecipazione e, in particolare, non creano quote indivise di partecipazione trasmissibili a terzi. Le quote o contributi associativi sono intrasmissibili per causa di morte.
	\end{enumerate}
	
	\part{Disposizioni finali}
	\section{Trasformazione - terzo settore}
	\begin{enumerate}
		\item L'assemblea, a maggioranza assoluta dei presenti, potrà deliberare la trasformazione dell'Associazione in Società sportiva di capitali o cooperativa sportiva.
		\item L'assemblea ordinaria potrà deliberare l'iscrizione al registro unico nazionale del terzo settore.
	\end{enumerate}
	
	\section{Scioglimento}
	\begin{enumerate}
		\item Lo scioglimento dell'Associazione è deliberato dall'Assemblea ai sensi degli articoli 11.1.b e 12.4 del presente Statuto.
		\item L'Assemblea che delibera lo scioglimento nomina, preferibilmente tra i membri del  Consiglio Direttivo e/o  tra i membri del Collegio dei Revisori dei Conti , uno o più liquidatori.
		\item Esperita la liquidazione di tutti i beni mobili e immobili, estinte le obbligazioni in essere, il patrimonio residuo è devoluto a fini sportivi ai sensi dell'articolo, comma 1, lettera h), D.Lgs. 36/2021.
	\end{enumerate}
	
	\section{Clausola compromissoria}
	\begin{enumerate}
		\item Le controversie in materia sportiva saranno rimesse al collegio arbitrale previsto dai regolamenti della Federazione Italiana di appartenenza. A tal fine troveranno applicazione le norme sulla clausola compromissoria e sul collegio arbitrale previste dai vigenti regolamenti della Federazione e/o Ente di Promozione Sportiva di appartenenza.
	\end{enumerate}
	
	\section{Norma di rinvio}
	\begin{enumerate}
		\item Per quanto non espressamente contemplato nel presente Statuto, valgono, in quanto applicabili, le norme del codice civile e le disposizioni di legge vigenti ed emanande di settore, nonché le disposizioni delle Federazioni sportive Nazionali, Discipline Sportive Associate, Enti di Promozione Sportiva a cui l'Associazione era, è o sarà affiliata.
	\end{enumerate}
	
\end{document}